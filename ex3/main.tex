\documentclass[12pt]{article}
\usepackage{xeCJK}
\usepackage{caption}
\setCJKmainfont{KaiTi}
%\setmainfont{Times New Roman}
\setCJKfamilyfont{hei}{SimHei}                                    %黑体  hei
\newcommand{\hei}{\CJKfamily{hei}}                          % 黑体
\usepackage{amsmath}
\usepackage{amsthm}
\usepackage{tikz}
\usepackage{enumerate} 
\usepackage{fontspec}
\usepackage{diagbox}
\usepackage{amsfonts}

\newcommand{\numpy}{{\tt numpy}}    % tt font for numpy
\newcommand*{\dif}{\mathop{}\!\mathrm{d}}

\topmargin -.5in
\textheight 9in
\oddsidemargin -.25in
\evensidemargin -.25in
\textwidth 7in

\begin{document}
%\newfontfamily{\Hei}{SimHei}
% ========== Edit your name here
\author{罗雁天}
\title{习题课3}
\maketitle

\medskip

% ========== Begin answering questions here
\begin{enumerate}

%\item {\hei 再次考虑“匹配”问题。$n$个人随机地选取帽子,试问恰好戴上了自己帽子的人数的期望和方差是多少。}
%
%\begin{proof}[解]
%	带上自己帽子的人数$X$满足:
%	\begin{equation}
%	X=H_1+H_2+\cdots+H_n,\quad H_k=\left\{
%	\begin{array}{cc}
%	1 & \mbox{第$k$个人戴对帽子}\\
%	0 & \mbox{第$k$个人戴错帽子}
%	\end{array}
%	\right.
%	\end{equation}
%	则由\begin{equation}
%	P(H_k=1)=1-P(H_k=0)=\frac{1}{n}
%	\end{equation}
%	得到\begin{equation}
%	E(H_k)=1\times P(H_k=1)+0\times P(H_k=0)=\frac{1}{n}
%	\end{equation}
%	从而\begin{equation}
%	\begin{aligned}
%	E(X)&=E(H_1+H_2+\cdots+H_n)\\
%	&=E(H_1)+E(H_2)+\cdots+E(H_n) \\
%	&=n\times \frac{1}{n} \\
%	&=1
%	\end{aligned}
%	\end{equation}
%	即戴上自己帽子的人数的期望值为1。也就是说,平均只有一个人戴上自己的帽子。相比于上次习题课得到的人数分布的结果,这里给出的期望能够更加直观地表明恰好匹配(戴上自己的帽子)是较为困难的事情。
%	
%	接下来求方差,首先求$E(X^2)$如下:
%	\begin{equation}
%	\label{eq1}
%	\begin{aligned}
%	E(X^2)&=E\left((H_1+H_2+\cdots+H_n)^2\right) \\
%	&=E\left(\sum_{k=1}^{n}H_k^2+\sum_{i=1}^{n}\sum_{j\ne i}H_iH_j\right) \\
%	&=\sum_{k=1}^nE(H_k^2)+\sum_{i=1}^{n}\sum_{j\ne i}E(H_iH_j)
%	\end{aligned}
%	\end{equation}
%	根据我们的定义,我们可以得到:
%	\begin{equation}
%	E(H_k^2)=1\times P(H_k=1)+0\times P(H_k=0)=\frac{1}{n}
%	\end{equation}
%	$H_iH_j$表示第$i,j$两人均带对帽子,因此:
%	\begin{equation}
%	P(H_iH_j=1)=1-P(H_iH_j=0)=\frac{1}{n(n-1)}
%	\end{equation}
%	因此:
%	\begin{equation}
%	E(H_iH_j)=1\times P(H_iH_j=1)+0\times P(H_iH_j=0)=\frac{1}{n(n-1)}
%	\end{equation}
%	带入式(\ref{eq1})中得:
%	\begin{equation}
%	E(X^2)=n\times\frac{1}{n}+n(n-1)\times\frac{1}{n(n-1)}=2
%	\end{equation}
%	因此:
%	\begin{equation}
%	Var(X)=E(X^2)-\left[E(X)\right]^2=1
%	\end{equation}
%\end{proof}

\item {\hei n对夫妇共2n人待在一个房间内,现从中随机挑选m个人走出房间,问房间内还剩下的未被拆散的夫妇数目的均值。}
\begin{proof}[解]
	房间内剩下的未被拆散的夫妇数目$X$满足
	\begin{equation}
	X=C_1+C_2+\cdots+C_n,\quad C_k=\left\{
	\begin{array}{cc}
	1 & \mbox{第$k$对夫妇在房间}\\
	0 & \mbox{第$k$对夫妇不在房间}
	\end{array}
	\right.
	\end{equation}
	则由\begin{equation}
	P(C_k=1)=1-P(C_k=0)=\frac{\binom{2n-2}{m}}{\binom{2n}{m}}=\left(1-\frac{m}{2n}\right)\left(1-\frac{m}{2n-1}\right)
	\end{equation}
	得到\begin{equation}
	E(C_k)=1\times P(C_k=1)+0\times P(C_k=0)=\left(1-\frac{m}{2n}\right)\left(1-\frac{m}{2n-1}\right)
	\end{equation}
	从而\begin{equation}
	\begin{aligned}
	E(X)&=E(C_1+C_2+\cdots+C_n)\\
	&=E(C_1)+E(C_2)+\cdots+E(C_n) \\
	&=n\left(1-\frac{m}{2n}\right)\left(1-\frac{m}{2n-1}\right)
	\end{aligned}
	\end{equation}
\end{proof}

\item {\hei 某商场发行$n$种购物券,每一次在该商场购物,即可获得一张种类随机选取的购物券。该商场规定,如果能够收集齐所有的购物券,那么就可以得到该商场的大奖。问获得大奖所需要的购物次数的期望是多少。}

\begin{proof}[解]
	收集齐所有的购物券所需要的购物次数$X$可以写为
	\begin{equation}
	X=C_1+C_2+\cdots+C_n
	\end{equation}
	其中$C_k$表示在收集到$k-1$种购物券的前提下,收集到第$k$种购物券所需要的购物次数。由于手中已经有$k-1$种购物券,因此收集到新购物券的可能只有$n-k+1$种,所以$C_k$是几何分布随机变量,参数为$p=(n-k+1)/n$,因此我们有:
	\begin{equation}
	E(C_k)=\frac{1}{p}=\frac{n}{n-k+1}
	\end{equation}
	所以:
	\begin{equation}
	\begin{aligned}
	E(X)&=E(C_1+C_2+\cdots+C_n)\\
	&=E(C_1)+E(C_2)+\cdots+E(C_n) \\
	&=1+\frac{n}{n-1}+\frac{n}{n-2}+\cdots+\frac{n}{1} \\
	&=n\left(1+\frac{1}{2}+\cdots+\frac{1}{n}\right)
	\end{aligned}
	\end{equation}
	由微积分的知识可以知道$E(X)\approx n\ln n$,可见,当$n$较大时,如果想收集到所有的购物券,则需要付出数倍于购物券数目的购物次数(以30张购物券为例,$\ln(30)\approx3.4$) 。这正是商家的精明之处。
\end{proof}

\item {\hei 司机在一年发生事故的次数满足$\lambda$的泊松分布,而$\lambda$服从参数为$\mu$的指数分布,问某一司机上一年不发生事故,今年也不发生事故的概率。}
\begin{proof}[解]
	假设该司机在一年发生事故的次数为$X$,则有:
	\begin{equation}
	\begin{aligned}
	P(X=k)&=\int_{0}^{+\infty}P(X=k|\lambda=x)f_{\lambda}(x)dx \\
	&=\int_{0}^{+\infty}\frac{x^ke^{-x}}{k!}\mu e^{-\mu x}dx \\
	&=\frac{\mu}{k!}\int_{0}^{+\infty}x^ke^{-(\mu+1)x}dx \\
	&=\frac{\mu}{k!}\frac{\Gamma(k+1)}{(\mu+1)^{k+1}} \\
	&=\frac{\mu}{(\mu+1)^{k+1}}
	\end{aligned}
	\end{equation}
	因此,司机连续两年不发生事故的概率为(两年发生事故相互独立):
	\begin{equation}
	P=\left(\frac{\mu}{\mu+1}\right)^2
	\end{equation}
\end{proof}

\item {\hei 设$R$罐中有$n$个红球,$H$罐中有$n$个黑球。每次操作都从两个罐中各随机取出一球,交换放置($R$罐中取出的放入$H$罐,$H$罐中取出的放入$R$罐),问$k$次操作后,$R$罐中的红球数目的均值。}

\begin{proof}[解]
	$k$次操作后$R$罐的红球数目满足:
	\begin{equation}
	X=R_1+R_2+\cdots+R_n,\quad R_t=\left\{
	\begin{array}{cc}
	1 & \mbox{第$t$个红球仍在$R$罐}\\
	0 & \mbox{第$t$个红球不在$R$罐}
	\end{array}
	\right.
	\end{equation}
	对于操作开始前$R$罐中原有的每一个红球而言,其经过$k$次操作仍回到$R$罐意味着被选中的次数为偶数(从来未被选中意味着选中次数为0,仍为偶数)。所以:
	\begin{equation}
	P(R_t=1)=1-P(R_t=0)=\sum_{m=2i}\binom{k}{m}\left(\frac{1}{n}\right)^m\left(1-\frac{1}{n}\right)^{k-m}
	\end{equation}
	根据二项式定理,我们可以计算出偶数次项的和可以计算如下:
	\begin{equation}
	\sum_{m=2i}\binom{k}{m}a^mb^{k-m}=\frac{1}{2}\left(\left(b+a\right)^k+\left(b-a\right)^k\right)
	\end{equation}
	所以:
	\begin{equation}\begin{aligned}
	P(R_t=1)&=\frac{1}{2}\left(\left(1-\frac{1}{n}+\frac{1}{n}\right)^k+\left(1-\frac{1}{n}-\frac{1}{n}\right)^k\right) \\
	&=\frac{1}{2}\left(1+\left(1-\frac{2}{n}\right)^k\right) 
	\end{aligned}
	\end{equation}
	从而
	\begin{equation}
	\begin{aligned}
	E(X)&=E(R_1+R_2+\cdots+R_n) \\
	&=E(R_1)+E(R_2)+\cdots+E(R_n) \\
	&=\frac{n}{2}\left(1+\left(1-\frac{2}{n}\right)^k\right)
	\end{aligned}
	\end{equation}
\end{proof}

%\item {\hei 设随机变量$X\sim N(0,\sigma^2),Y=[X]$,即$Y$是$X$的整数部分(例如:$[1.2]=1,[-2.3]=-3$),计算$Y$的期望}
%\begin{proof}[解]
%	设$X$的分布函数为$f_X(x)$,$Y$是整数集上的离散随机变量,我们有:
%	\begin{equation}
%	P(Y=n)=\int_{n}^{n+1}f_X(x)dx
%	\end{equation}
%	考虑到$f_X(x)$是偶函数,所以:
%	\begin{equation}
%	\begin{aligned}
%	E(Y)&=\sum_{n=-\infty}^{+\infty}n\times P(y=n) \\
%	&=\sum_{n=-\infty}^{+\infty}n\times\int_{n}^{n+1}f_X(x)dx \\
%	&=\int_{1}^{2}f_X(x)dx+2\int_{2}^{3}f_X(x)dx+\cdots\\
%	&-\int_{0}^{1}f_X(x)dx-2\int_{1}^{2}f_X(x)dx-3\int_{2}^{3}f_X(x)dx-\cdots \\
%	&=-\int_{0}^{+\infty}f_X(x)dx \\
%	&=-\frac{1}{2}
%	\end{aligned}
%	\end{equation}
%\end{proof}
\item {\hei $X,Y$独立都服从$\mathcal{N}(0,1)$,求$E[(X-3Y)^2|2X+Y=3]$}
\begin{proof}[解]
	根据定义有:
	\begin{equation}
	E[(X-3Y)^2|2X+Y=3]=\frac{\iint_{2x+y=3}(x-3y)^2p(x,y)\dif x\dif y}{\iint_{2x+y=3}p(x,y)\dif x\dif y}
	\end{equation}
	首先计算分母如下:
	\begin{equation}
	\begin{aligned}
	\iint_{2x+y=3}p(x,y)\dif x\dif y&=\iint_{2x+y=3}p(x)p(y)\dif x\dif y \\
	&=\iint_{2x+y=3}\frac{1}{\sqrt{2\pi}}\exp\left(-\frac{x^2}{2}\right)\frac{1}{\sqrt{2\pi}}\exp\left(-\frac{y^2}{2}\right)\dif x\dif y \\
	&=\iint_{2x+y=3}\frac{1}{2\pi}\exp\left(-\frac{x^2+y^2}{2}\right)\dif x\dif y \\
	&\mbox{令}x=1+t,y=1-2t \\
	&=\frac{1}{2\pi}\int_{-\infty}^{+\infty}\exp\left[-\frac{(1+t)^2+(1-2t)^2}{2}\right]\dif t \\
	&=\frac{1}{2\pi}\int_{-\infty}^{+\infty}\exp\left[-\frac{5t^2-2t+2}{2}\right]\dif t \\
	&\mbox{令}x=t-\frac{1}{5}\\
	&=\exp\left(\frac{9}{10}\right)\cdot \frac{1}{2\pi}\int_{-\infty}^{+\infty}\exp\left[-\frac{5x^2}{2}\right]\dif t \\
	&=\exp\left(\frac{9}{10}\right)\cdot \frac{1}{\sqrt{10\pi}}
	\end{aligned}
	\end{equation}
	然后计算分子如下:
	\begin{equation}
	\begin{aligned}
	\iint_{2x+y=3}(x-3y)^2p(x,y)\dif x\dif y&=\iint_{2x+y=3}(x-3y)^2p(x,y)\dif x\dif y\\
	&=\frac{1}{2\pi}\iint_{2x+y=3}(x-3y)^2\exp\left(-\frac{x^2+y^2}{2}\right)\dif x\dif y \\
	&\mbox{令}x=1+t,y=1-2t \\
	&=\frac{1}{2\pi}\int_{-\infty}^{+\infty}(7t-2)^2\exp\left[-\frac{5t^2-2t+2}{2}\right]\dif t \\
	&\mbox{令}x=t-\frac{1}{5}\\
	&=\exp\left(\frac{9}{10}\right)\cdot \frac{1}{2\pi}\int_{-\infty}^{+\infty}(7x-\frac{3}{5})^2\exp\left[-\frac{5x^2}{2}\right]\dif t \\
	&=\exp\left(\frac{9}{10}\right)\cdot \frac{1}{\sqrt{10\pi}}\times\frac{254}{25}
	\end{aligned}
	\end{equation}
	所以:
	\begin{equation}
	E[(X-3Y)^2|2X+Y=3]=\frac{254}{25}
	\end{equation}
\end{proof}

\item {\hei 设有$k$种不同的优惠券,每次收集到第$i$种优惠券的概率为$p_i$,$\sum_{i=1}^kp_i=1$,且每次收集之间是相互独立的。如果收集了$n$张优惠券,那么优惠券的种类的期望是多少?}
\begin{proof}[解]
	优惠券的种类数$X$满足:
	\begin{equation}
	X=H_1+H_2+\cdots+H_k,\quad H_i=\left\{
	\begin{array}{cc}
	1 & \mbox{收集到第$i$种优惠券}\\
	0 & \mbox{没有收集到第$i$种优惠券}
	\end{array}
	\right.
	\end{equation}
	那么我们有:
	\begin{equation}
	P(H_i=1)=1-P(H_i=0)=1-(1-p_i)^n
	\end{equation}
	所以:
	\begin{equation}
	E(H_i)=1\times P(H_i=1)+0\times P(H_i=0)=1-(1-p_i)^n
	\end{equation}
	从而:
	\begin{equation}
	\begin{aligned}
		E(X)&=E(H_1+H_2+\cdots+H_k)\\
		&=E(H_1)+E(H_2)+\cdots+E(H_k) \\
		&=\sum_{i=1}^{k}\left(1-(1-p_i)^n\right)
	\end{aligned}
	\end{equation}
\end{proof}

%\item {\hei 在单位圆上随机取两个点构成一条弦,试计算从原点到该弦的距离所服从的概率密度函数}
%\begin{proof}[解]
%	设两点对圆心的张角为$\theta$,则$\theta\sim U(0,\pi)$,我们有:
%	\begin{equation}
%	f_\theta(\theta)=\frac{1}{\pi}
%	\end{equation}
%	设从原点到弦的距离为$r$,则$r=\cos\left(\frac{\theta}{2}\right)$。由于在$\theta\in[0,\pi],r\in[0,1]$时,$r$是$\theta$的单调函数,于是:
%	\begin{equation}
%	f_r(r)=f_\theta(\theta)\left|\frac{d\theta}{dr}\right|=\frac{2}{\pi\sqrt{1-r^2}},r\in[0,1]
%	\end{equation}
%	
%	{\hei 注意}:这里的随机取点认为是两者所对的圆心角(取范围$[0,\pi]$)服从均匀分布,若选用其他方式,例如弦中点到圆心的距离服从均匀分布,则会得到不同的结果,这与Bertrand悖论的道理是相同的,即不同的样本空间会导致不同的结果。
%\end{proof}

\item {\hei 假设$\lambda$服从以$\alpha,\beta$为参数的Gamma分布,即$\lambda\sim \Gamma(\alpha,\beta)$。在给定$\lambda$的条件下,$x$服从以$\lambda$为参数的Poisson的分布,即$x|\lambda\sim Poisson(\lambda)$。试问,在给定$x$的条件下,$\lambda$的分布是什么?}
\begin{proof}[解]
	由于 $\lambda \sim \Gamma(\alpha,\beta)$, 我们有:
	\begin{equation}
	p(\lambda)=\dfrac{\beta^{\alpha}\lambda^{\alpha-1}e^{-\beta\lambda}}{\Gamma(\alpha)}
	\end{equation}
	又由于 $x|\lambda \sim Poisson(\lambda)$, 因而:
	\begin{equation}
	p(x|\lambda)=\frac{\lambda^x}{x!}e^{-\lambda}
	\end{equation}
	所以,根据贝叶斯公式:
	\begin{equation}
	\begin{aligned}
	p(\lambda|x)&=\dfrac{p(x|\lambda)p(\lambda)}{p(x)} \\
	&=\dfrac{p(x|\lambda)p(\lambda)}{\int_{0}^{+\infty}p(x|\lambda)p(\lambda)\mathrm{d}\lambda} \\
	&=\dfrac{\frac{\lambda^x}{x!}e^{-\lambda}\cdot\dfrac{\beta^{\alpha}\lambda^{\alpha-1}e^{-\beta\lambda}}{\Gamma(\alpha)}}{\int_{0}^{+\infty}\frac{\lambda^x}{x!}e^{-\lambda}\cdot\dfrac{\beta^{\alpha}\lambda^{\alpha-1}e^{-\beta\lambda}}{\Gamma(\alpha)}\mathrm{d}\lambda} \\
	&=\dfrac{\lambda^{(\alpha+x-1)}e^{-(\beta+1)\lambda}}{\int_{0}^{+\infty}\lambda^{(\alpha+x-1)}e^{-(\beta+1)\lambda}\mathrm{d}\lambda} \\
	&=\dfrac{\lambda^{(\alpha+x-1)}e^{-(\beta+1)\lambda}}{\int_{0}^{+\infty}\left(\frac{u}{\beta+1}\right)^{(\alpha+x-1)}e^{-u}\mathrm{d}\left(\frac{u}{\beta+1}\right)} \qquad (u=(\beta+1)\lambda) \\
	&=\dfrac{\lambda^{(\alpha+x-1)}e^{-(\beta+1)\lambda}}{\frac{1}{(\beta+1)^{\alpha+x}}\int_{0}^{+\infty}u^{(\alpha+x-1)}e^{-u}\mathrm{d}u} \\
	&=\dfrac{(\beta+1)^{\alpha+x}\lambda^{(\alpha+x-1)}e^{-(\beta+1)\lambda}}{\Gamma(\alpha+x)}
	\end{aligned}
	\end{equation}
	所以, $\lambda|x \sim \Gamma(\lambda|\alpha+x,\beta+1)$
\end{proof}

\item {\hei 公交站起点站等可能发出$a,b$两班汽车,其中$a$停$m$站,$b$停$n$站,车上人数服从参数为$\lambda$的泊松分布,每名乘客在各站下车的概率相同,如果该站没有乘客下车,则公交车不停站。求一辆从起点站开出的公交车停站的期望。}
\begin{proof}[解]
	一辆从起点站开出的公交车停站的次数$X$的期望满足:
	\begin{equation}
	E(X)=\frac{1}{2}E(X_a)+\frac{1}{2}E(X_b)
	\end{equation}
	其中,$X_a$表示如果开出的是$a$车停站的次数,$X_b$表示如果开出的是$b$车停站的次数。对$a$车和$b$车的讨论类似,我们这里这讨论$a$车的情况。我们有:
	\begin{equation}
	X_a=X_{a1}+X_{a2}+\cdots+X_{am},\quad X_{at}=\left\{
	\begin{array}{cc}
	1 & \mbox{第$t$站有人下车}\\
	0 & \mbox{第$t$站没有人下车}
	\end{array}
	\right.
	\end{equation}
	由于人数$K$服从参数为$\lambda$的泊松分布,则有:
	\begin{equation}
	P(X_{at}=1|K=k)=1-P(X_{at}=0)=1-\left(1-\frac{1}{m}\right)^k
	\end{equation}
	所以:
	\begin{equation}
	\begin{aligned}
	E(X_{at}|K=k)&=1\times P(X_{at}=1|K=k)+0\times P(X_{at}=0|K=k) \\
	&=1-\left(1-\frac{1}{m}\right)^k
	\end{aligned}
	\end{equation}
	因此:
	\begin{equation}
	\begin{aligned}
	E(X_{at})&=E(E(X_{at}|K=k)) \\
	&=\sum_{k=0}^{+\infty}E(X_{at}|K=k)P(K=k) \\
	&=\sum_{k=0}^{+\infty}\left(1-\left(1-\frac{1}{m}\right)^k\right)\frac{\lambda^ke^{-\lambda}}{k!} \\
	&=1-e^{-\frac{\lambda}{m}}
	\end{aligned}
	\end{equation}
	所以我们可以得到:
	\begin{equation}
	\begin{aligned}
	E(X_a)&=E(X_{a1}+X_{a2}+\cdots+X_{am}) \\
	&=E(X_{a1})+E(X_{a2})+\cdots+E(X_{am}) \\
	&=m\left(1-e^{-\frac{\lambda}{m}}\right)
	\end{aligned}
	\end{equation}
	同理我们可以得到:
	\begin{equation}
	E(X_b)=n\left(1-e^{-\frac{\lambda}{n}}\right)
	\end{equation}
	所以:
	\begin{equation}
	E(X)=\frac{1}{2}\left(m\left(1-e^{-\frac{\lambda}{m}}\right)+n\left(1-e^{-\frac{\lambda}{n}}\right)\right)
	\end{equation}
\end{proof}

\item {\hei 设随机变量$X,Y$均服从均值为$0$,方差为$1$的正态分布,且相互独立,试求:$E\left[X^2+Y^2|\cos\left(\frac{X}{Y}\right)\right]$}
\begin{proof}[解]
	根据课后习题中的结论,我们有:
	\begin{equation}
	X^2+Y^2\mbox{与}\frac{X}{Y}\mbox{相互独立}
	\end{equation}
	因此,
	\begin{equation}
	X^2+Y^2\mbox{与}\cos\left(\frac{X}{Y}\right)\mbox{相互独立}
	\end{equation}
	所以:
	\begin{equation}
	\begin{aligned}
	E\left[X^2+Y^2|\cos\left(\frac{X}{Y}\right)\right]&=E\left[X^2+Y^2\right] \\
	&=E\left[X^2\right]+E\left[Y^2\right] \\
	&=2
	\end{aligned}
	\end{equation}
\end{proof}

\item {\hei 篮球赛时长$n$分钟,某球员每一分钟有一次投篮机会,投中概率为$p$,并且教练规定,若一次投篮不中,下一分钟不得投篮,需将机会交给队友,请计算一场球赛中该队员投中次数的均值}
\begin{proof}[解]
	该队员投中次数$X$满足
	\begin{equation}
	X=C_1+C_2+\cdots+C_n,\quad C_k=\left\{
	\begin{array}{cc}
	1 & \mbox{第$k$分钟投中}\\
	0 & \mbox{第$k$分钟未投或者未投中}
	\end{array}
	\right.
	\end{equation}
	引入随机变量$Y_k(k=1,2,\cdots,n)$:
	\begin{equation}
	Y_k=\left\{
	\begin{array}{cc}
	1 & \mbox{第$k$分钟投了}\\
	0 & \mbox{第$k$分钟未投}
	\end{array}
	\right.
	\end{equation}
	我们有:
	\begin{equation}
	\begin{aligned}
	P(Y_k=1)&=P(Y_k=1|Y_{k-1}=1)P(Y_{k-1}=1)+P(Y_k=1|Y_{k-1}=0)P(Y_{k-1}=0) \\
	&=p\times P(Y_{k-1}=1)+1\times P(Y_{k-1}=0) \\
	&=p\times P(Y_{k-1}=1)+1- P(Y_{k-1}=1)
	\end{aligned}
	\end{equation}
	由于$P(Y_1=1)=1$,求解上述递推式我们可以得到:
	\begin{equation}
	P(Y_k=1)=\frac{1}{2-p}\left[1-(p-1)^k\right]
	\end{equation}
	因此:
	\begin{equation}
	\begin{aligned}
	P(C_k=1)&=P(C_k=1|Y_k=1)P(Y_k=1)+P(C_k=1|Y_k=0)P(Y_k=0) \\
	&=p\times P(Y_k=1) + 0\times P(Y_k=0) \\
	&=\frac{p}{2-p}\left[1-(p-1)^k\right]
	\end{aligned}
	\end{equation}
	所以:
	\begin{equation}
	\begin{aligned}
	E(C_k)&=1\times P(C_k=1) + 0\times P(C_k=0) \\
	&=P(C_k=1) \\
	&=\frac{p}{2-p}\left[1-(p-1)^k\right]
	\end{aligned}
	\end{equation}
	从而\begin{equation}
	\begin{aligned}
	E(X)&=E(C_1+C_2+\cdots+C_n)\\
	&=E(C_1)+E(C_2)+\cdots+E(C_n) \\
	&=\sum_{k=1}^{n}\frac{p}{2-p}\left[1-(p-1)^k\right] \\
	&=\frac{np}{2-p}-\frac{p(p-1)}{(2-p)^2}+\frac{p(p-1)^{n+1}}{(2-p)^2}
	\end{aligned}
	\end{equation}
\end{proof}

\item {\hei 设$U_1,U_2,\cdots$为一相互独立的$(0,1)$上均匀分布的随机变量序列,求$E[N]$。其中:
\begin{equation*}
N=\min\left\{n:\sum_{i=1}^nU_i>1\right\}
\end{equation*}}
\begin{proof}[解]
	我们将得到一个更一般的结果。对于$x\in [0,1]$,令:
	\begin{equation}
	N(x)=\min\left\{n:\sum_{i=1}^nU_i>x\right\}
	\end{equation}
	再令:
	\begin{equation}
	m(x)=E[N(x)]
	\end{equation}
	即$N(x)$是部分和$\sum_{i=1}^{n}U_i$超过$x$的最小指标$n$,$m(x)$是$N(x)$的期望值。将$U_1$作为条件,我们有:
	\begin{equation}
	m(x)=E\left[E[N(x)|U_1=y]\right]=\int_{0}^{1}E[N(x)|U_1=y]dy
	\end{equation}
	对于条件期望$E[N(x)|U_1=y]$,我们有:
	\begin{equation}
	E[N(x)|U_1=y]=\left\{
	\begin{array}{cc}
	1, & y>x \\
	1+m(x-y), & y\le x
	\end{array}
	\right.
	\end{equation}
	代入上式可得:
	\begin{equation}
	\begin{aligned}
	m(x)&=1+\int_{0}^xm(x-y)dy \\
	&=1+\int_{0}^xm(u)du\qquad(\mbox{做变量替换}u=x-y)
	\end{aligned}
	\end{equation}
	对上式两边求微分可以得到:
	\begin{equation}
	m'(x)=m(x)
	\end{equation}
	求解上述微分方程可以得到:
	\begin{equation}
	m(x)=ke^x
	\end{equation}
	又由于$m(0)=1$可以得到$k=1$,因此:
	\begin{equation}
	m(x)=e^x
	\end{equation}
	所以,原问题$E[N]=m(1)=e$
\end{proof}

\item {\hei (平面上的随机徘徊)设在平面坐标系的原点放一质点,质点在平面上作如下的随机徘徊。
\begin{enumerate}[1)]
	\item 每一步质点移动一个单位的距离,且前进方向与$x$轴的夹角$\theta$在$(0,2\pi)$上均匀分布;
	\item 每一步质点移动一个单位的距离,前进方向只有上、下、左、右四种情况且概率相等。
\end{enumerate}
假设每秒运动一次,以第$n$秒时质点所在位置与原点的距离为半径画圆,请计算两种情况下圆面积的均值。}
\begin{proof}[解]
	用$(X_i,Y_i)$表示第$i$秒坐标的变化量,
	\begin{enumerate}[1)]
		\item 对于此问,我们有:
		\begin{equation}
		X_i=\cos \theta_i\qquad Y_i=\sin \theta_i
		\end{equation}
		其中$\theta_i,i=1,2,\cdots,n$为相互独立且在$(0,2\pi)$上均匀分布的随机变量。经过$n$秒之后,质点的位置为$\left(\sum_{i=1}^nX_i,\sum_{i=1}^nY_i\right)$,则质点所在位置与原点距离的平方$R^2$可以计算如下:
		\begin{equation}
		\begin{aligned}
		R^2&=\left(\sum_{i=1}^nX_i\right)^2+\left(\sum_{i=1}^nY_i\right)^2 \\
		&=\sum_{i=1}^n\left(X_i^2+Y_i^2\right)+\sum_{i=1}^n\sum_{j\ne i}\left(X_iX_j+Y_iY_j\right) \\
		&=n+\sum_{i=1}^n\sum_{j\ne i}\left(\cos \theta_i\cos \theta_j + \sin \theta_i\sin\theta_j\right)
		\end{aligned}
		\end{equation}
		由于$\theta_i与\theta_j$相互独立并且:
		\begin{equation}
		\begin{aligned}
		E[\cos\theta_i]&=\frac{1}{2\pi}\int_{0}^{2\pi}\cos\theta_i\dif \theta_i=0 \\
		E[\sin\theta_i]&=\frac{1}{2\pi}\int_{0}^{2\pi}\sin\theta_i\dif \theta_i=0
		\end{aligned}
		\end{equation}
		因此,对于圆面积$S=\pi R^2$,我们有:
		\begin{equation}
		E[S]=\pi E[R^2]=n\pi
		\end{equation}
		\item 对于此问,质点每一步的移动都是上、下、左、右四个方向之一,因此我们可以得到$X_i$和$Y_i$的联合分布为如表\ref{tab1}所示。
		\begin{table}[!h]
			\centering
			\caption{\label{tab1}$X,Y$的联合分布}
			\begin{tabular}{|c|c|c|c|}
				\hline
				\diagbox{X}{P}{Y} & 0 & 1 & -1 \\
				\hline
				0 & 0 & $\frac{1}{4}$ & $\frac{1}{4}$ \\
				\hline
				1 & $\frac{1}{4}$ & 0 & 0 \\
				\hline
				-1 & $\frac{1}{4}$ & 0 & 0 \\
				\hline
			\end{tabular}
		\end{table}
		
		经过$n$秒之后,质点的位置为$\left(\sum_{i=1}^nX_i,\sum_{i=1}^nY_i\right)$,则质点所在位置与原点距离的平方$R^2$可以计算如下:
		\begin{equation}
		\begin{aligned}
		R^2&=\left(\sum_{i=1}^nX_i\right)^2+\left(\sum_{i=1}^nY_i\right)^2 \\
		&=\sum_{i=1}^n\left(X_i^2+Y_i^2\right)+\sum_{i=1}^n\sum_{j\ne i}\left(X_iX_j+Y_iY_j\right) \\
		&=n+\sum_{i=1}^n\sum_{j\ne i}\left(X_iX_j+Y_iY_j\right)
		\end{aligned}
		\end{equation}
		下面我们计算$E[X_iX_j]$:
		\begin{equation}
		\begin{aligned}
		E[X_iX_j]&=0\times P(X_iX_j=0)+(-1)\times P(X_iX_j=-1)+1\times P(X_iX_j=1) \\
		&=0\times\frac{3}{4}+(-1)\times\frac{1}{8}+1\times\frac{1}{8} \\
		&=0
		\end{aligned}
		\end{equation}
		同理我们有$E[Y_iY_j]=0$,因此因此,对于圆面积$S=\pi R^2$,我们有:
		\begin{equation}
		E[S]=\pi E[R^2]=n\pi
		\end{equation}
	\end{enumerate}
	综上所述,两种情况下圆面积的均值均为$n\pi$
\end{proof}

\item {\hei 已知$(X_1,X_2)\sim \mathcal{N}(0,0,\sigma_1^2,\sigma_2^2,\rho)$,设$Y_1=\cos X_1,Y_2=\cos X_2$,试求$Y_1,Y_2$的协方差}
\begin{proof}[解]
	首先我们证明一个结论,方便之后的计算。对于任意的正态分布$X\sim\mathcal{N}(0,\sigma^2)$
	\begin{equation}
	\begin{aligned}
		g_X(t)=E(e^{jtX})&=\int_{-\infty}^{+\infty}e^{jtx}\frac{1}{\sqrt{2\pi}\sigma}e^{-\frac{x^2}{2\sigma^2}}\dif x \\
		&=\frac{1}{\sqrt{2\pi}\sigma}\int_{-\infty}^{+\infty}e^{-\frac{x^2-2\sigma^2jtx}{2\sigma^2}}\dif x \\
		&\mbox{令}y=x-\sigma^2jt \\
		&=e^{-\frac{1}{2}\sigma^2t^2}\cdot\frac{1}{\sqrt{2\pi}\sigma}\int_{-\infty}^{+\infty}e^{-\frac{y^2}{2\sigma^2}}\dif y \\
		&=e^{-\frac{1}{2}\sigma^2t^2}
	\end{aligned}
	\end{equation}
	利用此结论,计算$E[Y_1]$如下:
	\begin{equation}
	\label{eq1}
	\begin{aligned}
	E[Y_1]=E[\cos X_1]&=E\left[\frac{e^{jX_1}+e^{-jX_1}}{2}\right] \\
	&=\frac{1}{2}E\left[e^{jX_1}\right]+\frac{1}{2}E\left[e^{-jX_1}\right] \\
	&=\frac{1}{2}g_{X_1}(1)+\frac{1}{2}g_{X_1}(-1) \\
	&=\frac{1}{2}e^{-\frac{1}{2}\sigma_1^2}+\frac{1}{2}e^{-\frac{1}{2}\sigma_1^2} \\
	&=e^{-\frac{1}{2}\sigma_1^2}
	\end{aligned}
	\end{equation}
	同理我们可以得到$E[Y_2]$:
	\begin{equation}
	E[Y_2]=e^{-\frac{1}{2}\sigma_2^2}
	\end{equation}
	下面计算$E[Y_1Y_2]$:
	\begin{equation}
	\label{eq2}
	\begin{aligned}
	E[Y_1Y_2]&=E[\cos X_1\cos X_2] \\
	&=E\left[\frac{1}{2}\left(\cos(X_1+X_2)+\cos(X_1-X_2)\right)\right] \\
	&=\frac{1}{2}E\left[\cos(X_1+X_2)\right]+\frac{1}{2}E\left[\cos(X_1-X_2)\right]
	\end{aligned}
	\end{equation}
	可以证明:
	\begin{equation}
	\begin{aligned}
	X_1+X_2 &\sim \mathcal{N}(0,\sigma_1^2+\sigma_2^2+2\rho\sigma_1\sigma_2) \\
	X_1-X_2 &\sim \mathcal{N}(0,\sigma_1^2+\sigma_2^2-2\rho\sigma_1\sigma_2)
	\end{aligned}
	\end{equation}
	所以根据式(\ref{eq1})的计算方式可以得到:
	\begin{equation}
	\begin{aligned}
	E\left[\cos(X_1+X_2)\right]&=e^{-\frac{1}{2}(\sigma_1^2+\sigma_2^2+2\rho\sigma_1\sigma_2)} \\
	E\left[\cos(X_1-X_2)\right]&=e^{-\frac{1}{2}(\sigma_1^2+\sigma_2^2-2\rho\sigma_1\sigma_2)} \\
	\end{aligned}
	\end{equation}
	代入式(\ref{eq2})得:
	\begin{equation}
	E[Y_1Y_2]=\frac{1}{2}e^{-\frac{1}{2}(\sigma_1^2+\sigma_2^2+2\rho\sigma_1\sigma_2)}+\frac{1}{2}e^{-\frac{1}{2}(\sigma_1^2+\sigma_2^2-2\rho\sigma_1\sigma_2)}
	\end{equation}
	所以$Y_1,Y_2$的协方差为:
	\begin{equation}
	\begin{aligned}
	Cov(Y_1,Y_2)&=E[Y_1Y_2]-E[Y_1]E[Y_2] \\
	&=\frac{1}{2}e^{-\frac{1}{2}(\sigma_1^2+\sigma_2^2+2\rho\sigma_1\sigma_2)}+\frac{1}{2}e^{-\frac{1}{2}(\sigma_1^2+\sigma_2^2-2\rho\sigma_1\sigma_2)}-e^{-\frac{1}{2}\sigma_1^2}\cdot e^{-\frac{1}{2}\sigma_2^2} \\
	&=\frac{1}{2}e^{-\frac{1}{2}(\sigma_1^2+\sigma_2^2+2\rho\sigma_1\sigma_2)}+\frac{1}{2}e^{-\frac{1}{2}(\sigma_1^2+\sigma_2^2-2\rho\sigma_1\sigma_2)}-e^{-\frac{1}{2}(\sigma_1^2+\sigma_2^2)}
	\end{aligned}
	\end{equation}
\end{proof}


\end{enumerate}


\end{document}
