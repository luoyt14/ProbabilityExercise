\documentclass[12pt]{article}
\usepackage{xeCJK}
\usepackage{caption}
\setCJKmainfont{KaiTi}
%\setmainfont{Times New Roman}
\setCJKfamilyfont{hei}{SimHei}                                    %黑体  hei
\newcommand{\hei}{\CJKfamily{hei}}                          % 黑体
\usepackage{amsmath}
\usepackage{amsthm}
\usepackage{tikz}
\usepackage{enumerate} 
\usepackage{fontspec}
%\usepackage{setspace}

\newcommand{\numpy}{{\tt numpy}}    % tt font for numpy
%\renewcommand{\baselinestretch}{1.0}

\topmargin -.5in
\textheight 9in
\oddsidemargin -.25in
\evensidemargin -.25in
\textwidth 7in

\begin{document}
%\newfontfamily{\Hei}{SimHei}
% ========== Edit your name here
\author{闻健}
\title{习题课4}
\maketitle

\medskip

% ========== Begin answering questions here
\begin{enumerate}

%\item {\hei 设随机变量$X\sim N(0,\sigma^2),Y=[X]$,即$Y$是$X$向下取整所得(例如:$[1.2]=1,[-2.3]=-3$),计算$Y$的期望。}
%\begin{proof}[解]
%	设$X$的分布函数为$f_X(x)$,$Y$是整数集上的离散随机变量,我们有:
%	\begin{equation}
%	P(Y=n)=\int_{n}^{n+1}f_X(x)dx
%	\end{equation}
%	考虑到$f_X(x)$是偶函数,所以:
%	\begin{equation}
%	\begin{aligned}
%		E(Y)&=\sum_{n=-\infty}^{+\infty}n\times P(y=n) \\
%		&=\sum_{n=-\infty}^{+\infty}n\times\int_{n}^{n+1}f_X(x)dx \\
%		&=\int_{1}^{2}f_X(x)dx+2\int_{2}^{3}f_X(x)dx+\cdots\\
%		&-\int_{0}^{1}f_X(x)dx-2\int_{1}^{2}f_X(x)dx-3\int_{2}^{3}f_X(x)dx-\cdots \\
%		&=-\int_{0}^{+\infty}f_X(x)dx \\
%		&=-\frac{1}{2}
%	\end{aligned}
%	\end{equation}
%\end{proof}
%
%\item {\hei 在单位圆上随机取两个点构成一条弦,试计算从原点到该弦的距离所服从的概率密度函数。}
%\begin{proof}[解]
%	设两点对圆心的张角为$\theta$,则$\theta\sim U(0,\pi)$,我们有:
%	\begin{equation}
%	f_\theta(\theta)=\frac{1}{\pi}
%	\end{equation}
%	设从原点到弦的距离为$r$,则$r=\cos\left(\displaystyle{\frac{\theta}{2}}\right)$。由于在$\theta\in[0,\pi],r\in[0,1]$时,$r$是$\theta$的单调函数,于是:
%	\begin{equation}
%	f_r(r)=f_\theta(\theta)\left|\frac{d\theta}{dr}\right|=\frac{2}{\pi\sqrt{1-r^2}},r\in[0,1]
%	\end{equation}
%	
%	{\hei 注意}:这里的随机取点认为是两者所对的圆心角(取范围$[0,\pi]$)服从均匀分布,若选用其他方式,例如弦中点到圆心的距离服从均匀分布,则会得到不同的结果,这与Bertrand悖论的道理是相同的,即不同的样本空间会导致不同的结果。
%\end{proof}
%
%
%
%
%\item {\hei 某城市共有$N$辆车,车牌号从$1$到$N$($N$充分大),若随机地(可重复)记下$n$辆车的车牌号,其最大号码为$\xi$,求$E(\xi)$。}
%\begin{proof}[解]
%	\begin{equation}
%		P\left\{ \xi =m \right\}=\frac{{{m}^{n}}-{{\left( m-1 \right)}^{n}}}{{{N}^{n}}},m=1,2,\cdots ,N
%	\end{equation}
%	\begin{equation}
%		E\left( \xi  \right)=\sum\limits_{m=1}^{N}{m\cdot \frac{{{m}^{n}}-{{\left( m-1 \right)}^{n}}}{{{N}^{n}}}}=N-\sum\limits_{m=1}^{N-1}{\frac{{{m}^{n}}}{{{N}^{n}}}}
%	\end{equation}
%
%	当$N$充分大的时候,
%	\begin{equation}
%		\sum\limits_{m=1}^{N-1}{\frac{{{m}^{n}}}{{{N}^{n}}}}\text{=}N\sum\limits_{m=0}^{N-1}{\frac{1}{N}\frac{{{m}^{n}}}{{{N}^{n}}}\approx N\int_{0}^{1}{{{x}^{n}}dx}}=N\frac{1}{n+1}
%	\end{equation}
%	所以,
%	\begin{equation}
%		E\left( \xi  \right)=N-\sum\limits_{m=1}^{N-1}{\frac{{{m}^{n}}}{{{N}^{n}}}}\approx \frac{n}{n+1}N
%	\end{equation}
%\end{proof}
%
%
%\item{\hei 某海港每天早上对停泊船只供给净水,初始价为每吨$a$元,不够用续供则要加$50\%$的附加费;若用不完造成浪费则每吨加收资源费$a/4$元。设某轮船的净水用量是服从密度函数为$p\left( x \right)$的随机变量,为节约用水总开支,求该轮船的最佳首次供水量$y$。}
%\begin{proof}[解]
%	设实际用水量为$x$,总开支为$g\left( x \right)$,则有
%	\begin{equation}
%		\begin{aligned}
%			g\left( x \right)=\left\{ \begin{matrix}
%			ay+\displaystyle{\frac{a}{4}}\left( y-x \right),&x<y  \\
%			\\
%			ay+\displaystyle{\frac{3a}{2}}\left( x-y \right),&x\ge y  \\
%			\end{matrix} \right.
%		\end{aligned}
%	\end{equation}
%	于是,总开支的期望为
%	\begin{equation}
%		\begin{aligned}
%		f\left( y \right)&=E\left[ g\left( x \right) \right] \\ 
%		& =\int_{0}^{y}{\left( ay+\frac{a}{4}y-\frac{a}{4}x \right)p\left( x \right)dx}+\int_{y}^{+\infty }{\left( ay+\frac{3a}{2}x-\frac{3a}{2}y \right)p\left( x \right)dx} \\ 
%		& =\frac{5a}{4}y\int_{0}^{y}{p\left( x \right)dx}-\frac{a}{4}\int_{0}^{y}{xp\left( x \right)dx}+\frac{3a}{2}\int_{y}^{+\infty }{xp\left( x \right)dx}-\frac{a}{2}y\int_{y}^{+\infty }{p\left( x \right)dx}  
%		\end{aligned}
%	\end{equation}
%	为求期望的最小值,将上式对$y$求导
%	\begin{equation}
%		\begin{aligned}
%		 {f}'\left( y \right)&=\frac{5a}{4}\int_{0}^{y}{p\left( x \right)dx}+\frac{5a}{4}yp\left( y \right)-\frac{a}{4}yp\left( y \right)-\frac{3a}{2}yp\left( y \right)-\frac{a}{2}\int_{y}^{+\infty }{p\left( x \right)dx}+\frac{a}{2}yp\left( y \right) \\ 
%		& =\frac{5a}{4}\int_{0}^{y}{p\left( x \right)dx}-\frac{a}{2}\int_{y}^{+\infty }{p\left( x \right)dx} \\ 
%		& =\frac{7a}{4}\int_{0}^{y}{p\left( x \right)dx}-\frac{a}{2}  
%		\end{aligned}
%	\end{equation}
%	令$ {f}'\left( y \right)=0 $,则得最佳供水量$ y $应满足$\displaystyle{ \int_{0}^{y}{p\left( x \right)dx}= \frac{2}{7}}$。
%\end{proof}
%
%\item {\hei 已知随机变量$ X $服从参数为$ \lambda $的指数分布,求随机变量$ {Y_1}=1-{{e}^{-\lambda X}} $和$ {Y_2}={{e}^{-\lambda X}} $的概率分布。}
%\begin{proof}[解]
%	先考虑随机变量$ {Y_1} $,记$ Y=g(X)=1-{{e}^{-\lambda X}} $,易知$ g(x) $ 为关于$ x $的严格单调递增函数,值域为$ [0,1] $;它的反函数$ x=h(y)=\displaystyle{-\frac{\ln \left( 1-y \right)}{\lambda }} $也是单调递增函数,在$ y\in \left[ 0,1 \right] $上连续可导,导函数为
%	\begin{equation}
%		{h}'\left( y \right)=\displaystyle{\frac{1}{\lambda \left( 1-y \right)}}
%	\end{equation}
%	可见,$ {h}'\left( y \right)>0 $,所以$ \left| {h}'\left( y \right) \right|={h}'\left( y \right) $。因此,当$ 0\le y\le 1 $时,
%	\begin{equation}
%		\begin{aligned}
%		 {{p}_{Y}}\left( y \right)&={{p}_{X}}\left[ h\left( y \right) \right]\cdot\left|{h}'\left( y \right)\right| \\ 
%		& =\frac{1 }{\lambda \left( 1-y \right)}\lambda{{e}^{-\lambda \left[ -\frac{\ln \left( 1-y \right)}{\lambda } \right]}} \\ 
%		& =1  
%		\end{aligned}
%	\end{equation}
%	所以,随机变量$ Y $的概率密度函数为
%	\begin{equation}
%		\begin{aligned}
%			{{p}_{Y}}\left( y \right)=\left\{ \begin{matrix}
%			1,&0\le y\le 1  \\
%			0,&\text{others}  \\
%			\end{matrix} \right.
%		\end{aligned}
%	\end{equation}
%	即$ Y\sim U\left( 0,1 \right) $。随机变量$ {Y_2} $与之类似,也服从均匀分布,证明略。
%\end{proof}
%
%
%
%\item {\hei 设连续随机变量$ X $服从参数为$ \lambda $的指数分布,另一离散型随机变量$ Y $的可能取值为全体正整数,其分布列为$ {{P}_{Y}}\left( Y=k \right)={{P}_{X}}\left[ \left( k-1 \right)\Delta \le X\le k\Delta  \right] $,其中$ k=1,2,\cdots $,常数$ \Delta >0 $。那么,随机变量$ Y $服从哪种常用的概率分布。}
%\begin{proof}[解]
%	指数函数的分布函数为
%	\begin{equation}
%		\begin{aligned}
%			F\left( x \right)=\left\{ \begin{matrix}
%			1-{{e}^{-\lambda x}},&x\ge 0  \\
%			0,&x<0  \\
%			\end{matrix} \right.
%		\end{aligned}
%	\end{equation}
%	从而,$ Y $的概率分布为
%	\begin{equation}
%		\begin{aligned}
%			 {{P}_{Y}}\left( Y=k \right)& ={{P}_{X}}\left[ \left( k-1 \right)\Delta \le X\le k\Delta  \right] \\ 
%			& =F\left( k\Delta  \right)-F\left[ \left( k-1 \right)\Delta  \right] \\ 
%			& ={{\left( {{e}^{-\lambda \Delta }} \right)}^{k-1}}\left( 1-{{e}^{-\lambda \Delta }} \right)
%		\end{aligned}
%	\end{equation}
%	令$ p=1-{{e}^{-\lambda \Delta }} $,则易知$ 0<p<1 $,且有
%	\begin{equation}
%		{{P}_{Y}}\left( Y=k \right)=p{{\left( 1-p \right)}^{k-1}}
%	\end{equation}
%	所以,随机变量$ Y $服从几何分布,即$ Y\sim Ge\left( 1-{{e}^{-\lambda \Delta }} \right) $。
%\end{proof}
%
%
%\item {\hei 实验室器皿中产生甲、乙两类细菌的机会是相等的,且产生的细菌的总数服从参数为$ \lambda $的Poisson分布。求:(1)产生了甲类细菌但没有乙类细菌的概率;(2)在已知产生了细菌而且没有甲类细菌的条件下,有2个乙类细菌的概率。}
%\begin{proof}[解]
%	(1) 设产生了$ k(k>0) $个细菌为事件$ A_k $,产生的细菌全是甲类细菌没有乙类细菌为事件$ B $,则
%	\begin{equation}
%		\begin{aligned}
%		P\left( B \right)& =\sum\limits_{k=1}^{+\infty }{P\left( {{A}_{k}} \right)P\left( B\left| {{A}_{k}} \right. \right)} \\ 
%		& =\sum\limits_{k=1}^{+\infty }{\frac{{{\lambda }^{k}}}{k!}{{e}^{-\lambda }}C_{k}^{k}{{\left( \frac{1}{2} \right)}^{k}}} \\ 
%		& ={{e}^{-\lambda }}\left( {{e}^{\frac{\lambda }{2}}}-1 \right) \\ 
%		\end{aligned}
%	\end{equation}
%	(2) 因为产生甲、乙两类细菌的机会是相等的,故所求概率等于
%	\begin{equation}
%		\begin{aligned}
%			P\left( {{A}_{2}}\left| B \right. \right)& =\frac{P\left( {{A}_{2}} \right)P\left( B\left| {{A}_{2}} \right. \right)}{P\left( B \right)} \\ 
%			& =\frac{\frac{{{\lambda }^{2}}}{2!}{{e}^{-\lambda }}C_{2}^{2}{{\left( \frac{1}{2} \right)}^{2}}}{{{e}^{-\lambda }}\left( {{e}^{\frac{\lambda }{2}}}-1 \right)} \\ 
%			& =\frac{{{\lambda }^{2}}}{8\left( {{e}^{\frac{\lambda }{2}}}-1 \right)} \\ 
%		\end{aligned}
%	\end{equation}
%\end{proof}
%
%
%\item {\hei 设随机变量$ X $取值于$ [0,1] $,若$ P\left( x \leq X \leq y\right)  $只与长度$ y-x $有关(对一切$ 0\leq x  \leq y\leq1 $),求$ X $的概率分布。}
%\begin{proof}[解]
%	记$ P\left( x\leq X \leq y\right)=f\left( y - x\right)   $,则对$ x = 0 $,$ \forall y\in \left[ 0,1 \right]  $,有
%	\begin{equation}
%		P\left( 0\leq X \leq y\right)=f\left( y\right) 
%	\end{equation}
%	对$ \forall {y_1},{y_2} \in \left[ 0,1 \right] $且$ {y_1}<{y_2}  $,有
%	\begin{equation}
%		P\left( 0\leq X \leq {y_1}+{y_2} \right)=P\left( 0\leq X < {y_1} \right)+P\left( {y_1} \leq X \leq {y_1}+{y_2} \right)
%	\end{equation}
%	即有
%	\begin{equation}
%		f \left(  {y_1}+{y_2}  \right)  = f \left(   {y_1} \right)  + f \left( {y_2} \right) 
%	\end{equation}
%	因此
%	\begin{equation}
%		f \left( {y} \right)  = C y
%	\end{equation}
%	由$ f \left( 1 \right)  =  f \left( 1-0 \right)  =  P\left( 0 \leq X \leq 1 \right) = 1 $ 推得$ C = 1 $,所以$ f \left( {x} \right)  = x $,即
%	\begin{equation}
%			P\left( 0 \leq X \leq x \right)=f \left( x \right) , x \in \left[ 0,1 \right]
%	\end{equation}
%	因而$ X $服从均匀分布,即$ X \sim U\left( 0,1 \right) $。
%\end{proof}
%
%
%
%\item {\hei 设$ X $ 为伯努利试验中第一个游程(连续的成功或失败)的长,求$ X $ 的概率分布及其期望。}
%\begin{proof}[解]
%	设每次试验中成功的概率为$ p $,则失败的概率为$ 1-p $,则随机变量$ X $ 的概率分布为
%	\begin{equation}
%		P \left(  X = k \right) = p^k \left(  1 - p \right) + p {\left(  1 - p \right)}^k , k = 1,2,\cdots
%	\end{equation}
%	期望为
%	\begin{equation}
%		\begin{aligned}
%			E \left( X \right) & = \sum\limits_{k=1}^{+\infty }{\left[ k p^k \left(  1 - p \right) + kp {\left(  1 - p \right)}^k \right] }\\
%			& = \frac{p}{1-p} + \frac{1-p}{p}
%		\end{aligned}
%	\end{equation}
%\end{proof}


%\item {\hei 食品厂把印有水浒108将之一的画卡作为赠券装入某种儿童食品袋中,每袋一卡,打开包装袋后获得每张画卡的概率是相同的。求:(1)为收集其中的$ r $张画卡所需购买的食品袋数$ X_r $ 的数学期望;(2)为集齐水浒108将,平均要购买多少袋?}
%\begin{proof}[解]
%	(1)将收集齐第$ j-1 $张赠券之后到收集齐第$ j $张赠券时所购买的食品袋数记为$ Y_j $,易知
%	\begin{equation}
%		{X_r} = {Y_1} + {Y_2} + \cdots + {Y_r}
%	\end{equation}
%	且$ Y_j  $服从几何分布
%	\begin{equation}
%		P\left( {Y_i} = k \right) ={{\left( 1-\frac{N-i+1}{N} \right)}^{k-1}}\frac{N-i+1}{N},k=1,2,\cdots
%	\end{equation}
%	其中$ N $代表画卡的总张数,此题中$ N=108 $,那么
%	\begin{equation}
%		E\left( {{Y}_{i}} \right)=\frac{N}{N-j+1},j=1,2,\cdots ,r
%	\end{equation}
%	因此
%	\begin{equation}
%		\begin{aligned}
%			E\left( {{X}_{r}} \right)& =E\left( {{Y}_{1}} \right)+E\left( {{Y}_{2}} \right)+\cdots +E\left( {{Y}_{r}} \right) \\ 
%			& =N\left( \frac{1}{N}+\cdots +\frac{1}{N-r+1} \right) \\ 
%		\end{aligned}
%	\end{equation}
%	(2) 由上一问知
%	\begin{equation}
%		\begin{aligned}
%			E\left( {{X}_{108}} \right)& =108\times \left( 1+\frac{1}{2}+\cdots +\frac{1}{108} \right) \\ 
%			& \approx 108\times \left( \ln 108+0.5772 \right) \\ 
%			& \approx 568.01
%		\end{aligned}
%	\end{equation}
%	其中,$ 0.5772 $是欧拉常数的近似值。所以,要集齐所有画卡平均要购买569袋该食品。
%\end{proof}


%\item {\hei 蛤蛤蛤}
%\begin{proof}[解]
%	
%\end{proof}


\item {\hei 随机变量$ X $的特征函数为$ \varphi (t) $,通过构造随机变量来证明:$ {{\left| \varphi \left( t \right) \right|}^{2}} $也是特征函数。}
\begin{proof}[解]
	假设随机变量$ X_{1} $和$ X_{2} $,分别与$ X $和$ -X $同分布,并且相互独立,则随机变量$ X_{1}+X_{2} $的特征函数为$ {{\varphi }_{{{X}_{1}}+{{X}_{2}}}}\left( t \right)=\varphi \left( t \right)\cdot \varphi \left( -t \right)=\varphi \left( t \right)\cdot \overline{\varphi \left( t \right)}={{\left| \varphi \left( t \right) \right|}^{2}} $。
\end{proof}


\item {\hei 设$ \varphi(t) $表示独立同分布随机变量序列$ \left\{ {{X}_{k}} \right\} $的特征函数,$ {\xi}  $为与$ \left\{ {{X}_{k}} \right\} $独立的随机变量,它的概率分布为$ P\left\{ \xi =n \right\}={{p}_{n}} $,其中$ n $为正整数。求随机变量$ Y=\sum\limits_{k=1}^{\xi }{{{X}_{k}}} $的特征函数。}
\begin{proof}[解]
	\begin{equation}
		\begin{aligned}
			{{\varphi }_{Y}}\left( t \right)&=E\left( {{\text{e}}^{itY}} \right) \\ 
			& =E\left[ E\left( {{\text{e}}^{itY}}\left| \xi  \right. \right) \right] \\ 
			& =\sum\limits_{n=1}^{\infty }{E\left[ {{\text{e}}^{it\left( {{X}_{1}}+{{X}_{2}}+\cdots +{{X}_{n}} \right)}}\left| \xi =n \right. \right]P\left\{ \xi =n \right\}} \\ 
			& =\sum\limits_{n=1}^{\infty }{E\left[ {{\text{e}}^{it\left( {{X}_{1}}+{{X}_{2}}+\cdots +{{X}_{n}} \right)}} \right]P\left\{ \xi =n \right\}} \\ 
			& =\sum\limits_{n=1}^{\infty }{{{\varphi }^{n}}\left( t \right){{p}_{n}}}
		\end{aligned}
	\end{equation}
\end{proof}


\item {\hei 设$ \left\{ {{X}_{n}} \right\} $为独立的随机变量序列,且具有相同的特征函数$ \varphi \left( t \right)=1+iat+o\left( t \right) $,其中$ a $为常数。证明:$ \displaystyle{\frac{1}{n}}\sum\limits_{k=1}^{n}{{{X}_{k}}}\stackrel{P}{\longrightarrow} a $。}
\begin{proof}[解]
	相同的特征函数意味着序列$ \left\{ {{X}_{n}} \right\} $服从相同的分布,且
	\begin{equation}
		E\left( {{X}_{n}} \right)=\frac{{\varphi }'\left( 0 \right)}{i}=a<\infty,
	\end{equation}
	故依据辛钦大数定律,对任意的$ \varepsilon>0 $,
	\begin{equation}
		\underset{n\to \infty }{\mathop{\lim }}\,P\left\{ \left| \frac{1}{n}\sum\limits_{k=1}^{n}{{{X}_{k}}}-\frac{1}{n}\sum\limits_{k=1}^{n}{E\left( {{X}_{k}} \right)} \right|<\varepsilon  \right\}=\underset{n\to \infty }{\mathop{\lim }}\,P\left\{ \left| \frac{1}{n}\sum\limits_{k=1}^{n}{{{X}_{k}}}-a \right|<\varepsilon  \right\}=1,
	\end{equation}
	即$ \displaystyle{\frac{1}{n}}\sum\limits_{k=1}^{n}{{{X}_{k}}} $依概率收敛于$ a $。
\end{proof}


%\item {\hei 利用特征函数证明教材98页的定理2.4.1。}
%\begin{proof}[解]
%	事件A发生的次数为$ X \sim b\left( n,{{p}_{n}} \right) $,其特征函数为
%	\begin{equation}
%		{{\varphi }_{n}}\left( t \right)={{\left[ 1+{{p}_{n}}\left( {{\text{e}}^{it}}-1 \right) \right]}^{n}}={{\left[ 1+\frac{n{{p}_{n}}\left( {{\text{e}}^{it}}-1 \right)}{n} \right]}^{n}},
%	\end{equation}
%	当$ n \longrightarrow \infty $时,则
%	\begin{equation}
%		\underset{n\to \infty }{\mathop{\lim }}\,{{\varphi }_{n}}\left( t \right)={{\left[ 1+\frac{\lambda \left( {{\text{e}}^{it}}-1 \right)}{n} \right]}^{n}}={{\text{e}}^{\lambda \left( {{\text{e}}^{it}}-1 \right)}},
%	\end{equation}
%	由唯一性定理可知泊松定理成立。
%\end{proof}


\item {\hei 独立随机变量序列$ \left\{ {{X}_{k}} \right\} $,其中$ k $为正整数,服从参数为$ {{k}^{r}} $的泊松分布。证明:当$ r < 1 $时,$ \left\{ {{X}_{k}} \right\} $服从大数定律。}
\begin{proof}[解]
	泊松分布的方差为$ \operatorname{var}\left( {{X}_{k}} \right)={{k}^{r}} $,当$ r < 1 $时,
	\begin{equation}
		\frac{1}{{{n}^{2}}}\operatorname{Var}\left( \sum\limits_{k=1}^{n}{{{X}_{k}}} \right)=\frac{1}{{{n}^{2}}}\sum\limits_{k=1}^{n}{{{k}^{r}}}\le \frac{n}{{{n}^{2}}}{{n}^{r}}={{n}^{r-1}} \longrightarrow 0,
	\end{equation}
	即马尔可夫条件成立,故$ \left\{ {{X}_{k}} \right\} $服从大数定律。
\end{proof}


\item {\hei 航空公司为了减小损失,会对一些航班进行超售管理。假设某航班的执飞飞机共有310个座位,根据历史数据,该航班的乘客不能按时登机的概率为0.03。那么如果航空公司实际售出314张票,求所有按时登机的乘客都有座位的概率。}
\begin{proof}[解]
	设按时登机的乘客数为$ Y_n $,则
	\begin{equation}
		Y_n \sim b(314,0.97), E(Y_n ) = 304.58, \operatorname{Var}\left( {{Y}_{n}} \right) = 9.1374 .
	\end{equation}
	所求概率为
	\begin{equation}
		P\left( {{Y}_{n}}\le 310 \right)\approx \Phi \left( \frac{310+0.5-304.58}{\sqrt{9.1374}} \right)\approx 0.9749 .
	\end{equation}
\end{proof}


\item {\hei 设$ n $重伯努利试验中,事件$ A $在每次试验中出现的概率为$ p\left( 0<p<1 \right) $,记$ X $为$ n $次试验中事件$ A $出现的次数。分别用切比雪夫不等式和中心极限定理估计满足
	\begin{equation}\nonumber
		P\left( \left| {\frac{X}{n}}-p \right|<{\frac{\sqrt{\operatorname{Var}\left( X \right)}}{2}} \right)>0.99
	\end{equation}
	的$ n $的值。}
\begin{proof}[解]
	(1)切比雪夫不等式:\\
	\begin{equation}
		P\left( \left| \frac{X}{n}-p \right|<\frac{\sqrt{\operatorname{Var}\left( X \right)}}{2} \right) \geq 1-\frac{\frac{\operatorname{Var}\left( X \right)}{{{n}^{2}}}}{{{\left( \frac{\sqrt{\operatorname{Var}\left( X \right)}}{2} \right)}^{2}}}=1-\frac{4}{{{n}^{2}}}>0.99,
	\end{equation}
	所以$ n > 20 $。\\
	(2)中心极限定理:\\
	\begin{equation}
		P\left( \left| \frac{X}{n}-p \right|<\frac{\sqrt{\operatorname{Var}\left( X \right)}}{2} \right)=P\left( \frac{\left| \frac{X}{n}-p \right|}{\frac{\sqrt{\operatorname{Var}\left( X \right)}}{n}}<\frac{n}{2} \right)=2\Phi \left( \frac{n}{2} \right)-1>0.99
	\end{equation}
	所以,$ \Phi \left( \displaystyle{\frac{n}{2}} \right)>0.995 $,则$ n>5.152 $。
\end{proof}


\item {\hei 试分别利用切比雪夫不等式和中心极限定理,求以不小于0.99的概率保证用频率估计事件发生概率的误差不大于0.01的独立伯努利试验次数。}
\begin{proof}[解]
	记事件发生次数为$ X $,所需试验次数为$ n $。因为事件发生概率未知,所以事件发生次数的方差应取最大值,即假设$ \operatorname{Var}\left( X \right)=npq=0.25n $。在此假设下,\\
	(1)切比雪夫不等式:\\
	\begin{equation}
		P\left( \left| \frac{X}{n}-p \right|<0.01 \right)\ge 1-\frac{\frac{0.25}{n}}{{{\left( 0.01 \right)}^{2}}}>0.99
	\end{equation}
	所以,$ n > 250000 $。\\
	(2)中心极限定理:\\
	\begin{equation}
		P\left( \left| \frac{X}{n}-p \right|<0.01 \right)=P\left( \frac{\left| \frac{X}{n}-p \right|}{\sqrt{\frac{0.25}{n}}}<\frac{0.01}{\sqrt{\frac{0.25}{n}}} \right)=2\Phi \left( \frac{0.01}{\sqrt{\frac{0.25}{n}}} \right)-1>0.99
	\end{equation}
	所以,$ \Phi \left( \displaystyle{\frac{0.01}{\sqrt{\frac{0.25}{n}}}} \right)>0.995 $,则$ n>16587.24 $。
\end{proof}


\item {\hei 设$ \left\{ {{X}_{n}} \right\} $为独立的随机变量序列,且$ P\left( {{X}_{n}}=\pm {{n}^{r}} \right)=\displaystyle{\frac{1}{2}} $,其中$ n $为正整数。利用中心极限定理证明:若$ r\geq  \displaystyle{\frac{1}{2}} $,$ \left\{ {{X}_{n}} \right\} $不服从大数定律。}
\begin{proof}[解]
	$  {{X}_{n}}  $的期望和方差分别为
	\begin{equation}
		E\left( {{X}_{n}} \right)=0,\operatorname{Var}\left( {{X}_{n}} \right)={{n}^{2r}}.
	\end{equation}
	对$ \delta > 0 $,
	\begin{equation}
		\begin{aligned}
			\frac{1}{B_{n}^{2+\delta }}\sum\limits_{k=1}^{n}{E\left( {{\left| {{X}_{k}}-{{\mu }_{k}} \right|}^{2+\delta }} \right)} &=\frac{\sum\limits_{k=1}^{n}{{{k}^{r\left( 2+\delta  \right)}}}}{{{\left( \sum\limits_{k=1}^{n}{{{k}^{2r}}} \right)}^{\frac{2+\delta }{2}}}} \\ 
			& =\frac{{{\left( 2r+1 \right)}^{\frac{2+\delta }{2}}}}{r\left( 2+\delta  \right)+1}\cdot \frac{{{n}^{r\left( 2+\delta  \right)+1}}}{{{n}^{\left( 2r+1 \right)\frac{2+\delta }{2}}}} \\ 
			& =\frac{{{\left( 2r+1 \right)}^{\frac{2+\delta }{2}}}}{r\left( 2+\delta  \right)+1}\cdot \frac{1}{{{n}^{\frac{\delta }{2}}}} \longrightarrow 0,
		\end{aligned}
	\end{equation}
	即$  {{X}_{n}}  $满足李雅普诺夫条件,所以可以使用中心极限定理。如果$ r\geq  \displaystyle{\frac{1}{2}} $,则对任意$ \varepsilon>0 $,有
	\begin{equation}
		\frac{{{\varepsilon }^{2}}{{n}^{2}}}{B_{n}^{2}}=\frac{{{\varepsilon }^{2}}{{n}^{2}}}{\sum\limits_{k=1}^{n}{{{k}^{2r}}}}\le \frac{{{\varepsilon }^{2}}{{n}^{2}}}{\sum\limits_{k=1}^{n}{k}}=\frac{2{{\varepsilon }^{2}}{{n}^{2}}}{n\left( n+1 \right)} \longrightarrow 2{{\varepsilon }^{2}}
	\end{equation}
	当$ n $充分大时,$ \displaystyle{\frac{\varepsilon n}{{{B}_{n}}}}<\sqrt{3}\varepsilon  $,且有
	\begin{equation}
		\begin{aligned}
			P\left\{ \frac{1}{n}\left| \sum\limits_{k=1}^{n}{{{X}_{k}}} \right|<\varepsilon  \right\} &=P\left\{ \frac{{{B}_{n}}}{n}\cdot \frac{1}{{{B}_{n}}}\left| \sum\limits_{k=1}^{n}{{{X}_{k}}} \right|<\varepsilon  \right\} \\ 
			& =P\left\{ \frac{1}{{{B}_{n}}}\left| \sum\limits_{k=1}^{n}{{{X}_{k}}} \right|<\frac{\varepsilon n}{{{B}_{n}}} \right\} \\ 
			& \le P\left\{ \frac{1}{{{B}_{n}}}\left| \sum\limits_{k=1}^{n}{{{X}_{k}}} \right|<\sqrt{3}\varepsilon  \right\}	.	
		\end{aligned}
	\end{equation}
	因此,
	\begin{equation}
		\begin{aligned}
			\underset{n\to \infty }{\mathop{\lim }}\,P\left\{ \frac{1}{n}\left| \sum\limits_{k=1}^{n}{{{X}_{k}}} \right|<\varepsilon  \right\}
			& \le \underset{n\to \infty }{\mathop{\lim }}\,P\left\{ \frac{1}{{{B}_{n}}}\left| \sum\limits_{k=1}^{n}{{{X}_{k}}} \right|<\sqrt{3}\varepsilon  \right\} \\ 
			& =\frac{1}{2\pi }\int_{-\sqrt{3}\varepsilon }^{\sqrt{3}\varepsilon }{{{\text{e}}^{-\frac{1}{2}{t}^{2}}}}\text{d}t \\ 
			& <1 .
		\end{aligned}
	\end{equation}
	故不服从大数定律。
\end{proof}


\item {\hei 用特征函数法直接证明棣莫弗-拉普拉斯中心极限定理。}
\begin{proof}[解]
	记$ {{S}_{n}}\sim b\left( n,p \right) $,则$ {{S}_{n}} $的特征函数为$ {{\varphi }_{n}}\left( t \right)={{\left( q+p{{\text{e}}^{it}} \right)}^{n}} $,由此$ Y_{n}^{*}=\displaystyle{\frac{{{S}_{n}}-np}{\sqrt{npq}}} $的特征函数为
	\begin{equation}
		\begin{aligned}
			{{f}_{n}}\left( t \right)&={{\left( q+p{{\text{e}}^{\frac{it}{\sqrt{npq}}}} \right)}^{n}}{{\text{e}}^{-\frac{npit}{\sqrt{npq}}}} \\ 
			& ={{\left( q{{\text{e}}^{-\frac{pit}{\sqrt{npq}}}}+p{{\text{e}}^{\frac{qit}{\sqrt{npq}}}} \right)}^{n}} \\ 
			& ={{\left[ q\left( 1-\frac{pit}{\sqrt{npq}}-\frac{p{{t}^{2}}}{2nq} \right)+p\left( 1+\frac{qit}{\sqrt{npq}}-\frac{q{{t}^{2}}}{2np} \right)+o\left( \frac{{{t}^{2}}}{n} \right) \right]}^{n}} \\ 
			& ={{\left[ 1-\frac{{{t}^{2}}}{2n}+o\left( \frac{{{t}^{2}}}{n} \right) \right]}^{n}}\longrightarrow {{\text{e}}^{-\frac{1}{2}{{t}^{2}}}}
		\end{aligned}
	\end{equation}
	由定理 4.2.6 即证得棣莫弗-拉普拉斯中心极限定理。
\end{proof}


%\item {\hei 蛤蛤蛤}
%\begin{proof}[解]
%	
%\end{proof}


%\item {\hei 蛤蛤蛤}
%\begin{proof}[解]
%	
%\end{proof}


%\item {\hei 蛤蛤蛤}
%\begin{proof}[解]
%	
%\end{proof}


%\item {\hei 蛤蛤蛤}
%\begin{proof}[解]
%	
%\end{proof}

\end{enumerate}




\end{document}
