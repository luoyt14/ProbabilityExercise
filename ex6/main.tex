\documentclass[12pt]{article}
\usepackage{xeCJK}
\usepackage{caption}
\setCJKmainfont{KaiTi}
%\setmainfont{Times New Roman}
\setCJKfamilyfont{hei}{SimHei}                                    %黑体  hei
\newcommand{\hei}{\CJKfamily{hei}}                          % 黑体
\usepackage{amsmath}
\usepackage{amsthm}
\usepackage{tikz}
\usepackage{enumerate} 
\usepackage{fontspec}
\usepackage{diagbox}
\usepackage{amsfonts}

\newcommand{\numpy}{{\tt numpy}}    % tt font for numpy
\newcommand*{\dif}{\mathop{}\!\mathrm{d}}

\topmargin -.5in
\textheight 9in
\oddsidemargin -.25in
\evensidemargin -.25in
\textwidth 7in

\begin{document}
%\newfontfamily{\Hei}{SimHei}
% ========== Edit your name here
\author{罗雁天}
\title{期末考题整理}
\maketitle

\medskip

% ========== Begin answering questions here
\begin{enumerate}
\item {\hei (15分)对于掷两颗骰子的随机试验。
\begin{enumerate}[(a)]
	\item 写出样本空间$\Omega$;
	\item 记事件$A$为点数之和是奇数,事件$D$为至少出现一个1点,求$P(A\cup D)$;
	\item 记事件$B$为某刻骰子出现奇数点,事件$C$为另一颗骰子出现奇数点。问$A,B,C$之间相互独立吗?说明理由。
\end{enumerate}
}
\begin{proof}[解]
	\begin{enumerate}[(a)]
	\item \begin{equation*}\Omega=N^2,N=\{1,2,3,4,5,6\}\end{equation*}
	\item \begin{equation*}
		\begin{aligned}
			&P(A)=\frac{|A|}{|\Omega|}=\frac{18}{36}=\frac{1}{2},P(D)=\frac{|D|}{|\Omega|}=\frac{11}{36} \\
			&P(AD)=\frac{|AD|}{|\Omega|}=\frac{6}{36}=\frac{1}{6} \\
			&\mbox{所以},P(A\cup D)=P(A)+P(D)-P(AD)=\frac{23}{36}
		\end{aligned}
	\end{equation*}
	\item 不独立。\begin{equation*}
		P(ABC)=P(\emptyset)=0\neq P(A)P(B)P(C)
	\end{equation*}
	\end{enumerate}
\end{proof}


\end{enumerate}


\end{document}
