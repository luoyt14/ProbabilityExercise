\documentclass[12pt]{article}
\usepackage{xeCJK}
\usepackage{caption}
\setCJKmainfont{KaiTi}
%\setmainfont{Times New Roman}
\setCJKfamilyfont{hei}{SimHei}                                    %黑体  hei
\newcommand{\hei}{\CJKfamily{hei}}                          % 黑体
\usepackage{amsmath}
\usepackage{amsthm}
\usepackage{tikz}
\usepackage{enumerate} 
\usepackage{fontspec}

\newcommand{\numpy}{{\tt numpy}}    % tt font for numpy

\topmargin -.5in
\textheight 9in
\oddsidemargin -.25in
\evensidemargin -.25in
\textwidth 7in

\begin{document}
%\newfontfamily{\Hei}{SimHei}
% ========== Edit your name here
\author{罗雁天}
\title{习题课1}
\maketitle

\medskip

% ========== Begin answering questions here
\begin{enumerate}

\item {\hei 在单位圆内随机挑选一条弦,请问弦长大于圆内接等边三角形边长的概率是多大。}
\begin{proof}[解]
	本题可以从三个方面考虑:
	\begin{itemize}
		\item 如果固定住弦的一个端点A,考察另外一个端点B,那么样本空间就是圆周,等概指的 是B在圆周上均匀分布。那么服从要求的B必然落在以A为端点的圆内接等边三角形中A的对 边所对应的劣弧中。这一段劣弧的长度恰为圆周长度的1/3,因此所求概率为1/3。
		\item 如果考察弦的中点O,以单位圆盘作为样本空间,等概指的是O在圆盘上均匀分布。那 么服从要求的O必然落在半径为1/2的单位圆的同心圆内。由于两个圆面积比为1/4,因此所 求概率为1/4。
		\item 同样考察弦的中点O,不过以与该弦垂直的半径作为样本空间,等概指的是O在该半径 上均匀分布。那么服从要求的O 必然落在靠近圆心的一半上。因此所求概率为1/2。
	\end{itemize}
	三种角度三个答案,看似矛盾实际却很合理。样本空间不同导致概率模型本身存在差 异,出现不同的结果也就不奇怪了。
\end{proof}

\item {\hei 假定某赌徒携带k元赌资进入赌场,赌博规则很简单,每赢一局则赢1元,否则输1元。假定每局赌博,赌徒赢的概率都是p,且各局间相互独立。试问,赌徒将所带赌资全部输光,被迫离开赌场的概率有多大?}
\begin{proof}[解]
	设事件$A_k$表示赌徒拥有$k$元初始赌本并最终输光,事件$W$表示赌徒赢得一局。那么有:
	\begin{equation}
	P(A_k)=P(A_k|W)P(W)+P(A_k|W^c)P(W^c)
	\end{equation}
	注意到$P(A_k|W)=P(A_{k+1}),P(A_k|W^c)=P(A_{k-1})$,我们有:
	\begin{equation}
	P(A_k)=pP(A_{k+1})+(1-p)P(A_{k-1})
	\end{equation}
	其中$p=P(W)$表示赌徒赢一局的概率。通过此递推式我们可以得到:
	\begin{equation}
	P(A_k)=a+b\left(\frac{1-p}{p}\right)^k
	\end{equation}
	其中$a,b$为确定性的参数,由初值决定。
	
	注意到$P(A_0)=1$,因此我们可以得到$a+b=1$;由于$0\le P(A_k)\le 1$。因此我们有如下结论:
	\begin{enumerate}[i)]
		\item 如果$p<0.5$(大多数赌场都满足这一条件),那么$b=0$,因此$P(A_k)\equiv 1$。这一点不难理解,如 果赌徒赢面小,那么输光应该是肯定的。
		\item 如果$p>0.5$(这种情况几乎不会出现),那么
		\begin{equation}
		P(A_k)=\left(\frac{1-p}{p}\right)^k+a\left(1-\left(\frac{1-p}{p}\right)^k\right)
		\end{equation}
		\item 如果$p=0.5$(赌场绝对公平),那么$P(A_k)\equiv1$。这一点
		很让人惊讶。即使在绝对公平的赌场内,赌徒输光也几乎是肯定的。
	\end{enumerate}
\end{proof}

\item {\hei 考虑医疗诊断问题,假设对于某种疾病,诊断的正确率为$p$。也就是说,如果就诊者确实患有该病,则医生能够以概率$p$做出准确诊断;如果就诊者实际没有患该病,医生做出正确判断的概率也是$p$。假设疾病自身的发病率是$q$。现已知某就诊者被医生诊断为患病,则其实际患该病的概率是多少?从中能得到什么结论呢?}
\begin{proof}[解]
	设事件$A$表示就诊者实际患病,$D$表示就诊者被诊断为患病,那么由Bayesian公式:
	\begin{equation}
	P(A|D)=\frac{P(D|A)P(A)}{P(D)}=\frac{P(D|A)P(A)}{P(D|A)P(A)+P(D|A^c)P(A^c)}
	\end{equation}
	带入$P(A)=q,P(A^c)=1-q,P(D|A)=p,P(D|A^c)=1-p$得:
	\begin{equation}
	P(A|D)=\frac{pq}{pq+(1-p)(1-q)}
	\end{equation}
	同理也可以得到:
	\begin{equation}
	P(A|D^c)=\frac{(1-p)q}{(1-p)q+p(1-q)}
	\end{equation}
	下面我们通过一些有趣的计算来进一步认识这一问题。如果疾病的发病率很低($q=0.01$),医生的医术值得信任,诊断的正确率很高($p=0.99$),那么在诊断患病的条件下,我们有:
	\begin{equation}
	P(A|D)=\frac{0.99\times 0.01}{0.99\times 0.01+0.01\times 0.99}=\frac{1}{2}
	\end{equation}
	也就是说,真实患病的概率只有$50\%$。不难发现,条件概率的计算中,疾病本身较低的发病率使得医生判断的准确性大打折扣。尝试将发病率$q$升高至0.5(例如普通感冒),那么在医生医术保持不变的情况下,有:
	\begin{equation}
	P(A|D)=\frac{0.99\times 0.5}{0.99\times 0.5+0.01\times 0.5}=0.99
	\end{equation}
	可见,对于普通常见病,医生的水平能够得到充分的体现。但如果进一步降低发病率$(q=0.001)$,即所谓“疑难杂症”,如果维持医生水平不变,那么:
	\begin{equation}
	P(A|D)=\frac{0.99\times 0.001}{0.99\times 0.001+0.01\times 0.999}=0.09
	\end{equation}
	诊断的实际准确率连$10\%$都不到,误报(实际没病,诊断有病)的概率达到了$90\%$。看起来,即使是非常称职的医生,当面临疑难杂症的时候也难免误报,患者应给予充分的理解,并在得知诊断结果之后保持冷静,争取用复诊的方法进一步确定是否患病。另一方面:
	\begin{equation}
	P(A|D^c)=\frac{0.01\times 0.001}{0.01\times 0.001+0.99\times 0.999}\approx10^{-5}\ll 1
	\end{equation}
	这说明,尽管误报的概率很高,但是漏报(实际患病,诊断无病)的概率却很低。所以,如果医生真的非常称职且水平很高,那么漏报性质的误诊率是能够充分降低的,即使面对的是十分罕见的疑难杂症。
\end{proof}
\end{enumerate}

\end{document}