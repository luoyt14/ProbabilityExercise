\documentclass[12pt]{article}
\usepackage{xeCJK}
\usepackage{caption}
\setCJKmainfont{KaiTi}
%\setmainfont{Times New Roman}
\setCJKfamilyfont{hei}{SimHei}                                    %黑体  hei
\newcommand{\hei}{\CJKfamily{hei}}                          % 黑体
\usepackage{amsmath}
\usepackage{amsthm}
\usepackage{tikz}
\usepackage{enumerate} 
\usepackage{fontspec}

\newcommand{\numpy}{{\tt numpy}}    % tt font for numpy

\topmargin -.5in
\textheight 9in
\oddsidemargin -.25in
\evensidemargin -.25in
\textwidth 7in

\begin{document}
%\newfontfamily{\Hei}{SimHei}
% ========== Edit your name here
\author{罗雁天}
\title{习题课2}
\maketitle

\medskip

% ========== Begin answering questions here
\begin{enumerate}

\item {\hei 再次考虑“匹配”问题。$n$个人随机地选取帽子,试问恰好戴上了自己帽子的人数的期望和方差是多少。}

\begin{proof}[解]
	带上自己帽子的人数$X$满足:
	\begin{equation}
	X=H_1+H_2+\cdots+H_n,\quad H_k=\left\{
	\begin{array}{cc}
	1 & \mbox{第$k$个人戴对帽子}\\
	0 & \mbox{第$k$个人戴错帽子}
	\end{array}
	\right.
	\end{equation}
	则由\begin{equation}
	P(H_k=1)=1-P(H_k=0)=\frac{1}{n}
	\end{equation}
	得到\begin{equation}
	E(H_k)=1\times P(H_k=1)+0\times P(H_k=0)=\frac{1}{n}
	\end{equation}
	从而\begin{equation}
	\begin{aligned}
	E(X)&=E(H_1+H_2+\cdots+H_n)\\
	&=E(H_1)+E(H_2)+\cdots+E(H_n) \\
	&=n\times \frac{1}{n} \\
	&=1
	\end{aligned}
	\end{equation}
	即戴上自己帽子的人数的期望值为1。也就是说,平均只有一个人戴上自己的帽子。相比于上次习题课得到的人数分布的结果,这里给出的期望能够更加直观地表明恰好匹配(戴上自己的帽子)是较为困难的事情。
	
	接下来求方差,首先求$E(X^2)$如下:
	\begin{equation}
	\label{eq1}
	\begin{aligned}
	E(X^2)&=E\left((H_1+H_2+\cdots+H_n)^2\right) \\
	&=E\left(\sum_{k=1}^{n}H_k^2+\sum_{i=1}^{n}\sum_{j\ne i}H_iH_j\right) \\
	&=\sum_{k=1}^nE(H_k^2)+\sum_{i=1}^{n}\sum_{j\ne i}E(H_iH_j)
	\end{aligned}
	\end{equation}
	根据我们的定义,我们可以得到:
	\begin{equation}
	E(H_k^2)=1\times P(H_k=1)+0\times P(H_k=0)=\frac{1}{n}
	\end{equation}
	$H_iH_j$表示第$i,j$两人均带对帽子,因此:
	\begin{equation}
	P(H_iH_j=1)=1-P(H_iH_j=0)=\frac{1}{n(n-1)}
	\end{equation}
	因此:
	\begin{equation}
	E(H_iH_j)=1\times P(H_iH_j=1)+0\times P(H_iH_j=0)=\frac{1}{n(n-1)}
	\end{equation}
	带入\ref{eq1}得:
	\begin{equation}
	E(X^2)=n\times\frac{1}{n}+n(n-1)\times\frac{1}{n(n-1)}=2
	\end{equation}
	因此:
	\begin{equation}
	Var(X)=E(X^2)-\left[E(X)\right]^2=1
	\end{equation}
\end{proof}

\item {\hei n对夫妇共2n人呆在一个房间内,现从中随机挑选m个人走出房间,问房间内还剩下的未被拆散的夫妇数目的均值。}
\begin{proof}[解]
	房间内剩下的未被拆散的夫妇数目$X$满足
	\begin{equation}
	X=C_1+C_2+\cdots+C_n,\quad C_k=\left\{
	\begin{array}{cc}
	1 & \mbox{第$k$对夫妇在房间}\\
	0 & \mbox{第$k$对夫妇不在房间}
	\end{array}
	\right.
	\end{equation}
	则由\begin{equation}
	P(C_k=1)=1-P(C_k=0)=\frac{\binom{2n-2}{m}}{\binom{2n}{m}}=\left(1-\frac{m}{2n}\right)\left(1-\frac{m}{2n-1}\right)
	\end{equation}
	得到\begin{equation}
	E(C_k)=1\times P(C_k=1)+0\times P(C_k=0)=\left(1-\frac{m}{2n}\right)\left(1-\frac{m}{2n-1}\right)
	\end{equation}
	从而\begin{equation}
	\begin{aligned}
	E(X)&=E(C_1+C_2+\cdots+C_n)\\
	&=E(C_1)+E(C_2)+\cdots+E(C_n) \\
	&=n\times \left(1-\frac{m}{2n}\right)\left(1-\frac{m}{2n-1}\right)
	\end{aligned}
	\end{equation}
\end{proof}

\item {\hei 某商场发行$n$种购物券,每一次在该商场购物,即可获得一张种类随机选取的购物券。该商场规定,如果能够收集齐所有的购物券,那么就可以得到该商场的大奖。问获得大奖所需要的购物次数的期望是多少。}

\begin{proof}[解]
	收集齐所有的购物券所需要的购物次数$X$可以写为
	\begin{equation}
	X=C_1+C_2+\cdots+C_n
	\end{equation}
	其中$C_k$表示在收集到$k-1$种购物券的前提下,收集到第$k$种购物券所需要的购物次数。由于手中已经有$k-1$种购物券,因此收集到新购物券的可能只有$n-k+1$种,所以$C_k$是几何分布随机变量,参数为$p=(n-k+1)/n$,因此我们有:
	\begin{equation}
	E(C_k)=\frac{1}{p}=\frac{n}{n-k+1}
	\end{equation}
	所以:
	\begin{equation}
	\begin{aligned}
	E(X)&=E(C_1+C_2+\cdots+C_n)\\
	&=E(C_1)+E(C_2)+\cdots+E(C_n) \\
	&=1+\frac{n}{n-1}+\frac{n}{n-2}+\cdots+\frac{n}{1} \\
	&=n\left(1+\frac{1}{2}+\cdots+\frac{1}{n}\right)
	\end{aligned}
	\end{equation}
	由微积分的知识可以知道$E(X)\approx n\ln n$,可见,当$n$较大时,如果想收集到所有的购物券,则需要付出数倍于购物券数目的购物次数(以30张购物券为例,$\ln(30)\approx3.4$) 。这正是商家的精明之处。
\end{proof}

\item {\hei 设$R$罐中有$n$个红球,$H$罐中有$n$个黑球。每次操作都从两个罐中各随机取出一球,交换放置($R$罐中取出的放入$H$罐,$H$罐中取出的放入$R$罐),问$k$次操作后,$R$罐中的红球数目的均值。}

\begin{proof}[解]
	$k$次操作后$R$罐的红球数目满足:
	\begin{equation}
	X=R_1+R_2+\cdots+R_n,\quad R_t=\left\{
	\begin{array}{cc}
	1 & \mbox{第$t$个红球仍在$R$罐}\\
	0 & \mbox{第$t$对夫妇不在$R$罐}
	\end{array}
	\right.
	\end{equation}
	对于操作开始前$R$罐中原有的每一个红球而言,其经过$k$次操作仍回到$R$罐意味着被选中的次数为偶数(从来未被选中意味着选中次数为0,仍为偶数)。所以:
	\begin{equation}
	P(R_t=1)=1-P(R_t=0)=\sum_{m=2i}\binom{k}{m}\left(\frac{1}{n}\right)^m\left(1-\frac{1}{n}\right)^{k-m}
	\end{equation}
	根据二项式定理,我们可以计算出偶数次项的和可以计算如下:
	\begin{equation}
	\sum_{m=2i}\binom{k}{m}a^mb^{k-m}=\frac{1}{2}\left(\left(b+a\right)^k+\left(b-a\right)^k\right)
	\end{equation}
	所以:
	\begin{equation}\begin{aligned}
	P(R_t=1)&=\frac{1}{2}\left(\left(1-\frac{1}{n}+\frac{1}{n}\right)^k+\left(1-\frac{1}{n}-\frac{1}{n}\right)^k\right) \\
	&=\frac{1}{2}\left(1+\left(1-\frac{2}{n}\right)^k\right) 
	\end{aligned}
	\end{equation}
	从而
	\begin{equation}
	\begin{aligned}
	E(X)&=E(R_1+R_2+\cdots+R_n) \\
	&=E(R_1)+E(R_2)+\cdots+E(R_n) \\
	&=\frac{n}{2}\left(1+\left(1-\frac{2}{n}\right)^k\right)
	\end{aligned}
	\end{equation}
\end{proof}

\item {\hei 设随机变量$X\sim N(0,\sigma^2),Y=[X]$,即$Y$是$X$的整数部分(例如:$[1.2]=1,[-2.3]=-3$),计算$Y$的期望}
\begin{proof}[解]
	设$X$的分布函数为$f_X(x)$,$Y$是整数集上的离散随机变量,我们有:
	\begin{equation}
	P(Y=n)=\int_{n}^{n+1}f_X(x)dx
	\end{equation}
	考虑到$f_X(x)$是偶函数,所以:
	\begin{equation}
	\begin{aligned}
	E(Y)&=\sum_{n=-\infty}^{+\infty}n\times P(y=n) \\
	&=\sum_{n=-\infty}^{+\infty}n\times\int_{n}^{n+1}f_X(x)dx \\
	&=\int_{1}^{2}f_X(x)dx+2\int_{2}^{3}f_X(x)dx+\cdots\\
	&-\int_{0}^{1}f_X(x)dx-2\int_{1}^{2}f_X(x)dx-3\int_{2}^{3}f_X(x)dx-\cdots \\
	&=-\int_{0}^{+\infty}f_X(x)dx \\
	&=-\frac{1}{2}
	\end{aligned}
	\end{equation}
\end{proof}

\item {\hei 设有$k$种不同的优惠券,每次收集到第$i$种优惠券的概率为$p_i$,$\sum_{i=1}^kp_i=1$,且每次收集之间是相互独立的。如果收集了$n$张优惠券,那么优惠券的种类的期望是多少?}
\begin{proof}[解]
	优惠券的种类数$X$满足:
	\begin{equation}
	X=H_1+H_2+\cdots+H_k,\quad H_i=\left\{
	\begin{array}{cc}
	1 & \mbox{收集到第$i$种优惠券}\\
	0 & \mbox{没有收集到第$i$种优惠券}
	\end{array}
	\right.
	\end{equation}
	那么我们有:
	\begin{equation}
	P(H_i=1)=1-P(H_i=0)=1-(1-p_i)^n
	\end{equation}
	所以:
	\begin{equation}
	E(H_i)=1\times P(H_i=1)+0\times P(H_i=0)=1-(1-p_i)^n
	\end{equation}
	从而:
	\begin{equation}
	\begin{aligned}
		E(X)&=E(H_1+H_2+\cdots+H_k)\\
		&=E(H_1)+E(H_2)+\cdots+E(H_k) \\
		&=\sum_{i=1}^{k}\left(1-(1-p_i)^n\right)
	\end{aligned}
	\end{equation}
\end{proof}

\item {\hei 在单位圆上随机取两个点构成一条弦,试计算从原点到该弦的距离所服从的概率密度函数}
\begin{proof}[解]
	设两点对圆心的张角为$\theta$,则$\theta\sim U(0,\pi)$,我们有:
	\begin{equation}
	f_\theta(\theta)=\frac{1}{\pi}
	\end{equation}
	设从原点到弦的距离为$r$,则$r=\cos\left(\frac{\theta}{2}\right)$。由于在$\theta\in[0,\pi],r\in[0,1]$时,$r$是$\theta$的单调函数,于是:
	\begin{equation}
	f_r(r)=f_\theta(\theta)\left|\frac{d\theta}{dr}\right|=\frac{2}{\pi\sqrt{1-r^2}},r\in[0,1]
	\end{equation}
	
	{\hei 注意}:这里的随机取点认为是两者所对的圆心角(取范围$[0,\pi]$)服从均匀分布,若选用其他方式,例如弦中点到圆心的距离服从均匀分布,则会得到不同的结果,这与Bertrand悖论的道理是相同的,即不同的样本空间会导致不同的结果。

\end{proof}
\end{enumerate}


\end{document}
