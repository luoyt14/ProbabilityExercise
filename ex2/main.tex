\documentclass[12pt]{article}
\usepackage{xeCJK}
\usepackage{caption}
\setCJKmainfont{KaiTi}
%\setmainfont{Times New Roman}
\setCJKfamilyfont{hei}{SimHei}                                    %黑体  hei
\newcommand{\hei}{\CJKfamily{hei}}                          % 黑体
\usepackage{amsmath}
\usepackage{amsthm}
\usepackage{tikz}
\usepackage{enumerate} 
\usepackage{fontspec}
%\usepackage{setspace}

\newcommand{\numpy}{{\tt numpy}}    % tt font for numpy
%\renewcommand{\baselinestretch}{1.0}

\topmargin -.5in
\textheight 9in
\oddsidemargin -.25in
\evensidemargin -.25in
\textwidth 7in

\begin{document}
%\newfontfamily{\Hei}{SimHei}
% ========== Edit your name here
\author{闻健}
\title{习题课2}
\maketitle

\medskip

% ========== Begin answering questions here
\begin{enumerate}

\item {\hei 设随机变量$X\sim N(0,\sigma^2),Y=[X]$,即$Y$是$X$向下取整所得(例如:$[1.2]=1,[-2.3]=-3$),计算$Y$的期望。}
\begin{proof}[解]
	设$X$的分布函数为$f_X(x)$,$Y$是整数集上的离散随机变量,我们有:
	\begin{equation}
	P(Y=n)=\int_{n}^{n+1}f_X(x)dx
	\end{equation}
	考虑到$f_X(x)$是偶函数,所以:
	\begin{equation}
	\begin{aligned}
		E(Y)&=\sum_{n=-\infty}^{+\infty}n\times P(y=n) \\
		&=\sum_{n=-\infty}^{+\infty}n\times\int_{n}^{n+1}f_X(x)dx \\
		&=\int_{1}^{2}f_X(x)dx+2\int_{2}^{3}f_X(x)dx+\cdots\\
		&-\int_{0}^{1}f_X(x)dx-2\int_{1}^{2}f_X(x)dx-3\int_{2}^{3}f_X(x)dx-\cdots \\
		&=-\int_{0}^{+\infty}f_X(x)dx \\
		&=-\frac{1}{2}
	\end{aligned}
	\end{equation}
\end{proof}

\item {\hei 在单位圆上随机取两个点构成一条弦,试计算从原点到该弦的距离所服从的概率密度函数。}
\begin{proof}[解]
	设两点对圆心的张角为$\theta$,则$\theta\sim U(0,\pi)$,我们有:
	\begin{equation}
	f_\theta(\theta)=\frac{1}{\pi}
	\end{equation}
	设从原点到弦的距离为$r$,则$r=\cos\left(\displaystyle{\frac{\theta}{2}}\right)$。由于在$\theta\in[0,\pi],r\in[0,1]$时,$r$是$\theta$的单调函数,于是:
	\begin{equation}
	f_r(r)=f_\theta(\theta)\left|\frac{d\theta}{dr}\right|=\frac{2}{\pi\sqrt{1-r^2}},r\in[0,1]
	\end{equation}
	
	{\hei 注意}:这里的随机取点认为是两者所对的圆心角(取范围$[0,\pi]$)服从均匀分布,若选用其他方式,例如弦中点到圆心的距离服从均匀分布,则会得到不同的结果,这与Bertrand悖论的道理是相同的,即不同的样本空间会导致不同的结果。
\end{proof}




\item {\hei 某城市共有$N$辆车,车牌号从$1$到$N$($N$充分大),若随机地(可重复)记下$n$辆车的车牌号,其最大号码为$\xi$,求$E(\xi)$。}
\begin{proof}[解]
	\begin{equation}
		P\left\{ \xi =m \right\}=\frac{{{m}^{n}}-{{\left( m-1 \right)}^{n}}}{{{N}^{n}}},m=1,2,\cdots ,N
	\end{equation}
	\begin{equation}
		E\left( \xi  \right)=\sum\limits_{m=1}^{N}{m\cdot \frac{{{m}^{n}}-{{\left( m-1 \right)}^{n}}}{{{N}^{n}}}}=N-\sum\limits_{m=1}^{N-1}{\frac{{{m}^{n}}}{{{N}^{n}}}}
	\end{equation}

	当$N$充分大的时候,
	\begin{equation}
		\sum\limits_{m=1}^{N-1}{\frac{{{m}^{n}}}{{{N}^{n}}}}\text{=}N\sum\limits_{m=0}^{N-1}{\frac{1}{N}\frac{{{m}^{n}}}{{{N}^{n}}}\approx N\int_{0}^{1}{{{x}^{n}}dx}}=N\frac{1}{n+1}
	\end{equation}
	所以,
	\begin{equation}
		E\left( \xi  \right)=N-\sum\limits_{m=1}^{N-1}{\frac{{{m}^{n}}}{{{N}^{n}}}}\approx \frac{n}{n+1}N
	\end{equation}
\end{proof}


\item{\hei 某海港每天早上对停泊船只供给净水,初始价为每吨$a$元,不够用续供则要加$50\%$的附加费;若用不完造成浪费则每吨加收资源费$a/4$元。设某轮船的净水用量是服从密度函数为$p\left( x \right)$的随机变量,为节约用水总开支,求该轮船的最佳首次供水量$y$。}
\begin{proof}[解]
	设实际用水量为$x$,总开支为$g\left( x \right)$,则有
	\begin{equation}
		\begin{aligned}
			g\left( x \right)=\left\{ \begin{matrix}
			ay+\displaystyle{\frac{a}{4}}\left( y-x \right),&x<y  \\
			\\
			ay+\displaystyle{\frac{3a}{2}}\left( x-y \right),&x\ge y  \\
			\end{matrix} \right.
		\end{aligned}
	\end{equation}
	于是,总开支的期望为
	\begin{equation}
		\begin{aligned}
		f\left( y \right)&=E\left[ g\left( x \right) \right] \\ 
		& =\int_{0}^{y}{\left( ay+\frac{a}{4}y-\frac{a}{4}x \right)p\left( x \right)dx}+\int_{y}^{+\infty }{\left( ay+\frac{3a}{2}x-\frac{3a}{2}y \right)p\left( x \right)dx} \\ 
		& =\frac{5a}{4}y\int_{0}^{y}{p\left( x \right)dx}-\frac{a}{4}\int_{0}^{y}{xp\left( x \right)dx}+\frac{3a}{2}\int_{y}^{+\infty }{xp\left( x \right)dx}-\frac{a}{2}y\int_{y}^{+\infty }{p\left( x \right)dx}  
		\end{aligned}
	\end{equation}
	为求期望的最小值,将上式对$y$求导
	\begin{equation}
		\begin{aligned}
		 {f}'\left( y \right)&=\frac{5a}{4}\int_{0}^{y}{p\left( x \right)dx}+\frac{5a}{4}yp\left( y \right)-\frac{a}{4}yp\left( y \right)-\frac{3a}{2}yp\left( y \right)-\frac{a}{2}\int_{y}^{+\infty }{p\left( x \right)dx}+\frac{a}{2}yp\left( y \right) \\ 
		& =\frac{5a}{4}\int_{0}^{y}{p\left( x \right)dx}-\frac{a}{2}\int_{y}^{+\infty }{p\left( x \right)dx} \\ 
		& =\frac{7a}{4}\int_{0}^{y}{p\left( x \right)dx}-\frac{a}{2}  
		\end{aligned}
	\end{equation}
	令$ {f}'\left( y \right)=0 $,则得最佳供水量$ y $应满足$\displaystyle{ \int_{0}^{y}{p\left( x \right)dx}= \frac{2}{7}}$。
\end{proof}

\item {\hei 已知随机变量$ X $服从参数为$ \lambda $的指数分布,求随机变量$ {Y_1}=1-{{e}^{-\lambda X}} $和$ {Y_2}={{e}^{-\lambda X}} $的概率分布。}
\begin{proof}[解]
	先考虑随机变量$ {Y_1} $,记$ Y=g(X)=1-{{e}^{-\lambda X}} $,易知$ g(x) $ 为关于$ x $的严格单调递增函数,值域为$ [0,1] $;它的反函数$ x=h(y)=\displaystyle{-\frac{\ln \left( 1-y \right)}{\lambda }} $也是单调递增函数,在$ y\in \left[ 0,1 \right] $上连续可导,导函数为
	\begin{equation}
		{h}'\left( y \right)=\displaystyle{\frac{1}{\lambda \left( 1-y \right)}}
	\end{equation}
	可见,$ {h}'\left( y \right)>0 $,所以$ \left| {h}'\left( y \right) \right|={h}'\left( y \right) $。因此,当$ 0\le y\le 1 $时,
	\begin{equation}
		\begin{aligned}
		 {{p}_{Y}}\left( y \right)&={{p}_{X}}\left[ h\left( y \right) \right]\cdot\left|{h}'\left( y \right)\right| \\ 
		& =\frac{1 }{\lambda \left( 1-y \right)}\lambda{{e}^{-\lambda \left[ -\frac{\ln \left( 1-y \right)}{\lambda } \right]}} \\ 
		& =1  
		\end{aligned}
	\end{equation}
	所以,随机变量$ Y $的概率密度函数为
	\begin{equation}
		\begin{aligned}
			{{p}_{Y}}\left( y \right)=\left\{ \begin{matrix}
			1,&0\le y\le 1  \\
			0,&\text{others}  \\
			\end{matrix} \right.
		\end{aligned}
	\end{equation}
	即$ Y\sim U\left( 0,1 \right) $。随机变量$ {Y_2} $与之类似,也服从均匀分布,证明略。
\end{proof}



\item {\hei 设连续随机变量$ X $服从参数为$ \lambda $的指数分布,另一离散型随机变量$ Y $的可能取值为全体正整数,其分布列为$ {{P}_{Y}}\left( Y=k \right)={{P}_{X}}\left[ \left( k-1 \right)\Delta \le X\le k\Delta  \right] $,其中$ k=1,2,\cdots $,常数$ \Delta >0 $。那么,随机变量$ Y $服从哪种常用的概率分布。}
\begin{proof}[解]
	指数函数的分布函数为
	\begin{equation}
		\begin{aligned}
			F\left( x \right)=\left\{ \begin{matrix}
			1-{{e}^{-\lambda x}},&x\ge 0  \\
			0,&x<0  \\
			\end{matrix} \right.
		\end{aligned}
	\end{equation}
	从而,$ Y $的概率分布为
	\begin{equation}
		\begin{aligned}
			 {{P}_{Y}}\left( Y=k \right)& ={{P}_{X}}\left[ \left( k-1 \right)\Delta \le X\le k\Delta  \right] \\ 
			& =F\left( k\Delta  \right)-F\left[ \left( k-1 \right)\Delta  \right] \\ 
			& ={{\left( {{e}^{-\lambda \Delta }} \right)}^{k-1}}\left( 1-{{e}^{-\lambda \Delta }} \right)
		\end{aligned}
	\end{equation}
	令$ p=1-{{e}^{-\lambda \Delta }} $,则易知$ 0<p<1 $,且有
	\begin{equation}
		{{P}_{Y}}\left( Y=k \right)=p{{\left( 1-p \right)}^{k-1}}
	\end{equation}
	所以,随机变量$ Y $服从几何分布,即$ Y\sim Ge\left( 1-{{e}^{-\lambda \Delta }} \right) $。
\end{proof}


\item {\hei 实验室器皿中产生甲、乙两类细菌的机会是相等的,且产生的细菌的总数服从参数为$ \lambda $的Poisson分布。求:(1)产生了甲类细菌但没有乙类细菌的概率;(2)在已知产生了细菌而且没有甲类细菌的条件下,有2个乙类细菌的概率。}
\begin{proof}[解]
	(1) 设产生了$ k(k>0) $个细菌为事件$ A_k $,产生的细菌全是甲类细菌没有乙类细菌为事件$ B $,则
	\begin{equation}
		\begin{aligned}
		P\left( B \right)& =\sum\limits_{k=1}^{+\infty }{P\left( {{A}_{k}} \right)P\left( B\left| {{A}_{k}} \right. \right)} \\ 
		& =\sum\limits_{k=1}^{+\infty }{\frac{{{\lambda }^{k}}}{k!}{{e}^{-\lambda }}C_{k}^{k}{{\left( \frac{1}{2} \right)}^{k}}} \\ 
		& ={{e}^{-\lambda }}\left( {{e}^{\frac{\lambda }{2}}}-1 \right) \\ 
		\end{aligned}
	\end{equation}
	(2) 所求概率为
	\begin{equation}
		\begin{aligned}
			P\left( {{A}_{2}}\left| B \right. \right)& =\frac{P\left( {{A}_{2}} \right)P\left( B\left| {{A}_{2}} \right. \right)}{P\left( B \right)} \\ 
			& =\frac{\frac{{{\lambda }^{2}}}{2!}{{e}^{-\lambda }}C_{2}^{2}{{\left( \frac{1}{2} \right)}^{2}}}{{{e}^{-\lambda }}\left( {{e}^{\frac{\lambda }{2}}}-1 \right)} \\ 
			& =\frac{{{\lambda }^{2}}}{8\left( {{e}^{\frac{\lambda }{2}}}-1 \right)} \\ 
		\end{aligned}
	\end{equation}
\end{proof}


\item {\hei 设随机变量$ X $取值于$ [0,1] $,若$ P\left( x \leq X \leq y\right)  $只与长度$ y-x $有关(对一切$ 0\leq x  \leq y\leq1 $),求$ X $的概率分布。}
\begin{proof}[解]
	记$ P\left( x\leq X \leq y\right)=f\left( y - x\right)   $,则对$ x = 0 $,$ \forall y\in \left[ 0,1 \right]  $,有
	\begin{equation}
		P\left( 0\leq X \leq y\right)=f\left( y\right) 
	\end{equation}
	对$ \forall {y_1},{y_2} \in \left[ 0,1 \right] $且$ {y_1}<{y_2}  $,有
	\begin{equation}
		P\left( 0\leq X \leq {y_1}+{y_2} \right)=P\left( 0\leq X < {y_1} \right)+P\left( {y_1} \leq X \leq {y_1}+{y_2} \right)
	\end{equation}
	即有
	\begin{equation}
		f \left(  {y_1}+{y_2}  \right)  = f \left(   {y_1} \right)  + f \left( {y_2} \right) 
	\end{equation}
	因此
	\begin{equation}
		f \left( {y} \right)  = C y
	\end{equation}
	由$ f \left( 1 \right)  =  f \left( 1-0 \right)  =  P\left( 0 \leq X \leq 1 \right) = 1 $ 推得$ C = 1 $,所以$ f \left( {x} \right)  = x $,即
	\begin{equation}
			P\left( 0 \leq X \leq x \right)=f \left( x \right) , x \in \left[ 0,1 \right]
	\end{equation}
	因而$ X $服从均匀分布,即$ X \sim U\left( 0,1 \right) $。
\end{proof}



\item {\hei 设$ X $ 为伯努利试验中第一个游程(连续的成功或失败)的长,求$ X $ 的概率分布及其期望。}
\begin{proof}[解]
	设每次试验中成功的概率为$ p $,则失败的概率为$ 1-p $,则随机变量$ X $ 的概率分布为
	\begin{equation}
		P \left(  X = k \right) = p^k \left(  1 - p \right) + p {\left(  1 - p \right)}^k , k = 1,2,\cdots
	\end{equation}
	期望为
	\begin{equation}
		\begin{aligned}
			E \left( X \right) & = \sum\limits_{k=1}^{+\infty }{\left[ k p^k \left(  1 - p \right) + kp {\left(  1 - p \right)}^k \right] }\\
			& = \frac{p}{1-p} + \frac{1-p}{p}
		\end{aligned}
	\end{equation}
\end{proof}


%\item {\hei 食品厂把印有水浒108将之一的画卡作为赠券装入某种儿童食品袋中,每袋一卡,打开包装袋后获得每张画卡的概率是相同的。求:(1)为收集其中的$ r $张画卡所需购买的食品袋数$ X_r $ 的数学期望;(2)为集齐水浒108将,平均要购买多少袋?}
%\begin{proof}[解]
%	(1)将收集齐第$ j-1 $张赠券之后到收集齐第$ j $张赠券时所购买的食品袋数记为$ Y_j $,易知
%	\begin{equation}
%		{X_r} = {Y_1} + {Y_2} + \cdots + {Y_r}
%	\end{equation}
%	且$ Y_j  $服从几何分布
%	\begin{equation}
%		P\left( {Y_i} = k \right) ={{\left( 1-\frac{N-i+1}{N} \right)}^{k-1}}\frac{N-i+1}{N},k=1,2,\cdots
%	\end{equation}
%	其中$ N $代表画卡的总张数,此题中$ N=108 $,那么
%	\begin{equation}
%		E\left( {{Y}_{i}} \right)=\frac{N}{N-j+1},j=1,2,\cdots ,r
%	\end{equation}
%	因此
%	\begin{equation}
%		\begin{aligned}
%			E\left( {{X}_{r}} \right)& =E\left( {{Y}_{1}} \right)+E\left( {{Y}_{2}} \right)+\cdots +E\left( {{Y}_{r}} \right) \\ 
%			& =N\left( \frac{1}{N}+\cdots +\frac{1}{N-r+1} \right) \\ 
%		\end{aligned}
%	\end{equation}
%	(2) 由上一问知
%	\begin{equation}
%		\begin{aligned}
%			E\left( {{X}_{108}} \right)& =108\times \left( 1+\frac{1}{2}+\cdots +\frac{1}{108} \right) \\ 
%			& \approx 108\times \left( \ln 108+0.5772 \right) \\ 
%			& \approx 568.01
%		\end{aligned}
%	\end{equation}
%	其中,$ 0.5772 $是欧拉常数的近似值。所以,要集齐所有画卡平均要购买569袋该食品。
%\end{proof}


%\item {\hei 蛤蛤蛤}
%\begin{proof}[解]
%	
%\end{proof}


\end{enumerate}




\end{document}
