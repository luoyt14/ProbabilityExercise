\documentclass[12pt]{article}
\usepackage{xeCJK}
\usepackage{caption}
\setCJKmainfont{KaiTi}
%\setmainfont{Times New Roman}
\setCJKfamilyfont{hei}{SimHei}                                    %黑体  hei
\newcommand{\hei}{\CJKfamily{hei}}                          % 黑体
\usepackage{amsmath}
\usepackage{amsthm}
\usepackage{tikz}
\usepackage{enumerate} 
\usepackage{fontspec}
\usepackage{diagbox}
\usepackage{amsfonts}

\newcommand{\numpy}{{\tt numpy}}    % tt font for numpy
\newcommand*{\dif}{\mathop{}\!\mathrm{d}}

\topmargin -.5in
\textheight 9in
\oddsidemargin -.25in
\evensidemargin -.25in
\textwidth 7in

\begin{document}
%\newfontfamily{\Hei}{SimHei}
% ========== Edit your name here
\author{罗雁天}
\title{习题整理}
\maketitle

\medskip

% ========== Begin answering questions here
\begin{enumerate}
\item {\hei $X,Y$相互独立,均服从参数为$\lambda$的指数分布,求$E[(X-Y)^2|X<Y]$. (15分)}
\begin{proof}[解]
	根据定义有:
	\begin{equation}
	E[(X-Y)^2|X<Y]=\frac{\iint_{x<y}(x-y)^2p(x,y)\dif x\dif y}{\iint_{x<y}p(x,y)\dif x\dif y}
	\end{equation}
	首先计算分母如下:
	\begin{equation}
	\begin{aligned}
	\iint_{x<y}p(x,y)\dif x\dif y&=\iint_{x<y}p(x)p(y)\dif x\dif y \\
	&=\iint_{x<y}\lambda e^{-\lambda x} \lambda e^{-\lambda y}\dif x\dif y \\
	&=\lambda^2 \iint_{x<y}e^{-\lambda x}e^{-\lambda y}\dif x\dif y \\
	&=\lambda^2 \int_{0}^{+\infty}\int_{0}^{y}e^{-\lambda x}e^{-\lambda y}\dif x\dif y \\
	&=\lambda^2 \int_{0}^{+\infty}e^{-\lambda y} \left(-\frac{1}{\lambda}\left(e^{-\lambda y}-1\right)\right)\dif y \\
	&=-\lambda\int_{0}^{+\infty}e^{-\lambda y}\left(e^{-\lambda y}-1\right)\dif y\\
	&=-\lambda\int_{0}^{+\infty}e^{-2\lambda y}\dif y+\lambda \int_{0}^{+\infty}e^{-\lambda y}\dif y \\
	&=-\frac{1}{2}+1 \\
	&=\frac{1}{2}
	\end{aligned}
	\end{equation}
	然后计算分子如下:
	\begin{equation}
	\begin{aligned}
	\iint_{x<y}(x-y)^2p(x,y)\dif x\dif y&=\iint_{x<y}(x-y)^2p(x)p(y)\dif x\dif y \\
	&=\iint_{x<y}(x-y)^2\lambda e^{-\lambda x} \lambda e^{-\lambda y}\dif x\dif y \\
	&=\lambda^2 \int_{0}^{+\infty}\int_{0}^{y}(x-y)^2e^{-\lambda x}e^{-\lambda y}\dif x\dif y \\
	&=\lambda^2 \int_{0}^{+\infty}e^{-\lambda y}\int_{0}^{y}(x-y)^2e^{-\lambda x}\dif x\dif y \\
	&\mbox{令}t=x-y \\ 
	&=\lambda^2 \int_{0}^{+\infty}e^{-\lambda y}\int_{-y}^{0}t^2e^{-\lambda t}e^{-\lambda y}\dif t\dif y \\
	&=\lambda^2 \int_{0}^{+\infty}e^{-\lambda y}\left(\frac{1}{\lambda}y^2-\frac{2}{\lambda^2}y-\frac{2}{\lambda^3}e^{-\lambda y}+\frac{2}{\lambda^3}\right) \dif y \\
	&=\int_{0}^{+\infty}y^2(\lambda e^{-\lambda y})\dif y-\frac{2}{\lambda}\int_{0}^{+\infty}y(\lambda e^{-\lambda y})\dif y\\
	&\quad-\frac{2}{\lambda}\int_{0}^{+\infty}(\lambda e^{-2\lambda y})\dif y+\frac{2}{\lambda}\int_{0}^{+\infty}(\lambda e^{-\lambda y})\dif y \\
	&=\frac{2}{\lambda^2}-\frac{2}{\lambda^2}-\frac{1}{\lambda^2}+\frac{2}{\lambda^2} \\
	&=\frac{1}{\lambda^2}
	\end{aligned}
	\end{equation}
	所以:
	\begin{equation}
	E[(X-Y)^2|X<Y]=\frac{2}{\lambda^2}
	\end{equation}
\end{proof}

\newpage
\item {\hei 考虑如下两个场景:\begin{enumerate}
		\item 不断抛掷骰子并读数,直到连续三次得到结果1,抛掷才能结束;
		\item 不断抛掷硬币并读数,直到出现如下序列时,抛掷才能结束;
		\begin{equation*}
		\mbox{正},\underbrace{\mbox{反},\mbox{反},\cdots,\mbox{反}}_k
		\end{equation*}
\end{enumerate}
假设从开始抛掷起到结束为止抛掷次数为$X$,求$E(X)$. (第一问10分,第二问10分)}

\begin{proof}[解]
	\begin{enumerate}
		\item 设从开始抛掷起到连续出现$k$次1时抛掷次数为$X_k$,给定$X_{k-1}$的情况下,我们考虑之后的一次抛掷,如果抛出1,那么游戏结束,此时$X_k=X_{k-1}+1$,如果抛出其他数字,那么相当于之前的努力都白费了,需要重新抛掷,此时$X_k=X_{k-1}+1+X_k$.因此我们有:
		\begin{equation}
		\begin{aligned}
		E(X_k)&=E(E(X_k|X_{k-1})) \\
		&=E\left[\frac{1}{6}(X_{k-1}+1)+\frac{5}{6}(X_{k-1}+1+X_{k})\right] \\
		&=E\left[X_{k-1}+1+\frac{5}{6}X_k\right] \\
		&=\frac{5}{6}E(X_k)+E(X_{k-1})+1
		\end{aligned}
		\end{equation}
		所以我们可以得到一个递推式:
		\begin{equation}
		E(X_k)=6E(X_{k-1})+6
		\end{equation}
		考虑$k=1$的情况,出现一次1就停止,即是几何分布,所以$E(X_1)=6$.所以求解上述递推式可以得到:
		\begin{equation}
		E(X_k)=\frac{6^{k+1}-6}{5}
		\end{equation}
		所以连续出现3次的就结束时:
		\begin{equation}
		E(X)=E(X_3)=\frac{6^4-6}{5}=258
		\end{equation}
		
		\item 设出现式(\ref{eq1})所表示的状态为$S_{n+1}(n=0,1,\cdots,k)$,$S_0$表示还没有开始扔骰子的状态。设$X_n$表示给定$S_{n}$的条件下,游戏结束时抛掷的次数,所以我们最后只需求$E(X_0)$
		\begin{equation}
		\label{eq1}
		\mbox{正},\underbrace{\mbox{反},\mbox{反},\cdots,\mbox{反}}_n
		\end{equation}
		
		\begin{itemize}
		\item 当$n=0$时,我们考虑第一次投掷,如果扔出正面朝上,那么$X_0=X_1+1$,反之,如果扔出反面朝上,我们则需要重新投掷,那么$X_0=X_0+1$,因此:
		\begin{equation}
		\begin{aligned}
		E(X_0)=E(E(X_0|X_1))&=E\left(\frac{1}{2}(X_1+1)+\frac{1}{2}(X_0+1)\right)\\
		&=\frac{1}{2}E(X_1+1)+\frac{1}{2}E(X_0+1)\\
		&=1+\frac{1}{2}E(X_0)+\frac{1}{2}E(X_1)
		\end{aligned}
		\end{equation}
		\item 当$n=1,2,\cdots,k$时,考虑下一次投掷,如果扔出反面朝上,那么$X_n=X_{n+1}+1$,反之,我们便回到了$S_1$的状态,有$X_n=X_1+1$,因此:
		\begin{equation}
		\begin{aligned}
		E(X_n)=E(E(X_n|X_{n+1}))&=E\left(\frac{1}{2}(X_{n+1}+1)+\frac{1}{2}(X_1+1)\right)\\
		&=\frac{1}{2}E(X_{n+1}+1)+\frac{1}{2}E(X_1+1)\\
		&=1+\frac{1}{2}E(X_{n+1})+\frac{1}{2}E(X_1)
		\end{aligned}
		\end{equation}
		\item 当$n=k+1$时,此时已经达到了游戏结束的条件,因此$E(X_{k+1})=0$
		\end{itemize}
	综上所述,我们从$E_{k+1}$开始向上递推,可以求出$E(X_0)=2^{k+1}$,即$E(X)=2^{k+1}$
	\end{enumerate}
\end{proof}

\newpage
\item {\hei 在一根长度为1的木棍上随机取一点A,若A点左边一段的长度小于$ \displaystyle{\frac{1}{3}} $,则在A点左边随机取一点B,否则在A右边随机取一点B,计算A、B两点间距离的均值和方差。}
\begin{proof}[解]
	以木棍左端点为坐标原点,设A点坐标为$ a $,用随机变量$ X $来表示B点可取的范围的长度,即
	\begin{equation}
	X=\left\{ \begin{matrix}
	a,&a<\displaystyle{\frac{1}{3}}  \\
	\\
	1-a,&a\ge \displaystyle{\frac{1}{3}}  \\
	\end{matrix} \right.
	\end{equation}
	由于$ a\sim U\left( 0,1 \right) $,所以
	\begin{equation}
	E\left( X \right)=\int_{0}^{1/3}{1\cdot a\text{d}a}+\int_{1/3}^{1}{1\cdot \left( 1-a \right)}\text{d}a=\frac{5}{18}
	\end{equation}
	\begin{equation}
	E\left( {{X}^{2}} \right)=\int_{0}^{1/3}{1\cdot {{a}^{2}}\text{d}a}+\int_{1/3}^{1}{1\cdot {{\left( 1-a \right)}^{2}}}\text{d}a=\frac{11}{81}
	\end{equation}
	设随机变量$ Y $表示A、B两点间的距离,则$ X $固定时,$ Y \sim U\left( 0,X \right) $,
	\begin{equation}
	E\left( Y \right)=E\left[ E\left( Y\left| X \right. \right) \right]=E\left( \int_{0}^{X}{y\cdot \frac{1}{X}\text{d}y} \right)=E\left( \frac{X}{2} \right)=\frac{5}{36}
	\end{equation}
	\begin{equation}
	E\left( {{Y}^{2}} \right)=E\left[ E\left( {{Y}^{2}}\left| X \right. \right) \right]=E\left( \int_{0}^{X}{{{y}^{2}}\cdot \frac{1}{X}\text{d}y} \right)=E\left( \frac{{{X}^{2}}}{3} \right)=\frac{11}{243}
	\end{equation}
	\begin{equation}
	\operatorname{Var}\left( Y \right)=E\left( {{Y}^{2}} \right)-{{\left[ E\left( Y \right) \right]}^{2}}=\frac{101}{3888}
	\end{equation}
	所以,A、B两点间距离的均值为$ \displaystyle{\frac{5}{36}} $,方差为$ \displaystyle{\frac{101}{3888}} $。
\end{proof}


\newpage
\item {\hei 设$ \left\{ {{X}_{k}} \right\} $为独立同分布的随机变量序列,其分布函数为$ {{F}_{X}}\left( x \right) $,设随机变量$ N(y) $满足
	\begin{equation}\nonumber
	N\left( y \right)=\min \left\{ k:{{X}_{k}}>y \right\}
	\end{equation}
	请计算
	\begin{equation}\nonumber
	\underset{y\to \infty }{\mathop{\lim }}\,P\left\{ N\left( y \right)\ge E\left[ N\left( y \right) \right] \right\}
	\end{equation}}
\begin{proof}[解]
	由题意知
	\begin{equation}
	\begin{aligned}
	P\left\{ N\left( y \right)=k \right\}&=P\left\{ {{X}_{1}}\le y,{{X}_{2}}\le y,\cdots ,{{X}_{k-1}}\le y,{{X}_{k}}>y \right\} \\ 
	& ={{\left[ {{F}_{X}}\left( y \right) \right]}^{k-1}}\left[ 1-{{F}_{X}}\left( y \right) \right]
	\end{aligned}
	\end{equation}
	即$ N\left( y \right)\sim Ge\left( p \right) $,其中$ p = 1-{{F}_{X}}\left( y \right) $,所以$ E\left[ N\left( y \right) \right]=\displaystyle{\frac{1}{p}} $,则
	\begin{equation}
	\begin{aligned}
	P\left\{ N\left( y \right)\ge E\left[ N\left( y \right) \right] \right\}&=\sum\limits_{k=\left\lfloor 1/p \right\rfloor +1}^{\infty }{{{\left( 1-p \right)}^{k-1}}p} \\ 
	& =p\sum\limits_{k=\left\lfloor 1/p \right\rfloor +1}^{\infty }{{{\left( 1-p \right)}^{k-1}}} \\ 
	& = p\cdot \frac{{{\left( 1-p \right)}^{\left\lfloor \frac{1}{p} \right\rfloor}}}{p} \\ 
	& ={{\left( 1-p \right)}^{\left\lfloor \frac{1}{p} \right\rfloor}}
	\end{aligned}
	\end{equation}
	其中$ \left\lfloor \cdotp  \right\rfloor $表示向下取整。当$ y\longrightarrow\infty $时,$ p\longrightarrow 0 $,所以
	\begin{equation}
	\underset{y\to \infty }{\mathop{\lim }}\,P\left\{ N\left( y \right)\ge E\left[ N\left( y \right) \right] \right\}=\underset{p\to 0}{\mathop{\lim }}\,{{\left( 1-p \right)}^{\left\lfloor \frac{1}{p} \right\rfloor}}
	\end{equation}
	而由于
	\begin{equation}
	{{\left( 1-p \right)}^{\frac{1}{p}}}\le {{\left( 1-p \right)}^{\left\lfloor \frac{1}{p} \right\rfloor }}<{{\left( 1-p \right)}^{\frac{1}{p}-1}}
	\end{equation}
	并且
	\begin{equation}
	\underset{p\to 0}{\mathop{\lim }}\,{{\left( 1-p \right)}^{\frac{1}{p}}}=\underset{p\to 0}{\mathop{\lim }}\,{{\left( 1-p \right)}^{\frac{1}{p}-1}}=\frac{1}{\text{e}}
	\end{equation}
	因此
	\begin{equation}
	\underset{y\to \infty }{\mathop{\lim }}\,P\left\{ N\left( y \right)\ge E\left[ N\left( y \right) \right] \right\} = \frac{1}{\text{e}}
	\end{equation}
\end{proof}

\end{enumerate}


\end{document}
